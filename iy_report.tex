% IY Report for Alexander Brown
\documentclass[a4paper,11pt]{report}

\usepackage{fullpage}
\usepackage{glossaries}
\usepackage{hyperref}

\newcommand{\HRule}{\rule{\linewidth}{0.5mm}}

\newacronym{ai}{AI}{Artificial Intelligence}
\newacronym{apar}{APAR}{Authorized Program Analysis Report}
\newacronym{api}{API}{Application Programming Interface}
\newacronym{apt}{APT}{APAR Polling Tool}

\newacronym{cics}{CICS}{Customer Information Control System}

\newacronym{db2}{DB2}{IBM DB2}
\newacronym{db2luw}{DB2 for LUW}{IBM DB2 for Linux, UNIX and Windows}

\newacronym{edt}{EDT}{Explorer Delivery Tool}

\newacronym{fmt}{FMT}{Fix Metadata Toolkit}
\newacronym{fv}{FV}{Functional Verification}

\newacronym{gbs}{GBS}{Global Business Servicess}
\newacronym{gnu}{GNU}{GNU's Not Unix}
\newacronym{gts}{GTS}{Global Technology Services}

\newacronym{ibm}{IBM}{International Business Machines Corporation}
\newacronym{it}{IT}{Industrial Trainee}
\newacronym{itd}{ITD}{Intergrated Technology Delivery}

\newacronym{j2ee}{Java EE}{Java Platform, Enterprise Edition}
\newacronym{json}{JSON}{JavaScript Object Notation}
\newacronym{jsp}{JSP}{JavaSever Pages} 

\newacronym{ldap}{LDAP}{Lightweight Directory Access Protocol}

\newacronym{mq}{MQ}{IBM WebSphere MQ}

\newacronym{osgi}{OSGi}{Open Services Gateway Initative framework}

\newacronym{pcomm}{PComm}{IBM Personal Communications}
\newacronym{pdm}{PDM}{Personal Development Manager}
\newacronym{pe}{PE}{Programming Error}
\newacronym{plx}{PL/X}{Programming Language/Cross System}
\newacronym{pmr}{PMR}{Problem Management Request}
\newacronym{pojo}{POJO}{Plain Old Java Object}
\newacronym{ptf}{PTF}{Program Temporary Fix}

\newacronym{retr}{RETAIN}{Remoth Technical Assistance Information Network}
\newacronym{rhel}{RHEL}{Red Hat Enterprise Linux}
\newacronym{rtc}{RTC}{Rational Team Concert}

\newacronym{scm}{SCM}{Source Code Management}
\newacronym{stg}{STG}{Systems and Technologies Group}
\newacronym{swg}{SWG}{Software Group}
\newacronym{swt}{SWT}{Standard Widget Toolkit}

\newacronym{uhl}{UHL}{Unit History Log}
\newacronym{uk}{UK}{United Kingdom}
\newacronym{us}{US}{United States of America}

\newacronym{was}{WAS}{IBM WebSphere Application Server}

\begin{document}

\begin{titlepage}

\begin{center}


% Upper part of the page

\textsc{\Large Industrial Year Report}\\[0.5cm]


% Title
\HRule \\[0.4cm]
{ \huge \bfseries IBM Level 3 CICS Service Engineer}\\[0.4cm]

\HRule \\[1.5cm]

% Author and supervisor
\begin{minipage}{0.4\textwidth}
\begin{flushleft} \large
\emph{Author:}\\
Alexander \textsc{Brown}
\end{flushleft}
\end{minipage}

\vfill

% Bottom of the page
{\large \today}

\end{center}

\end{titlepage}

\tableofcontents

\chapter{Organisational Environment}

\Gls{ibm} is an American multinational technology and consulting corporation, with it's headquarters
in New York, America. \Gls{ibm} sell a wide range of technical products, both hardware and
software, in areas ranging from mainframe computing to nanotechnology. As well as technical 
products \gls{ibm} also offer a range of consulting, hosting and infrastructure services.

\Gls{ibm} are also know for innovation; holding the largest number of \gls{us} patents, building
new technologies such as \gls{ibm} Watson and pushing corporate initiatives like Smart Planet. \\



\Gls{ibm} is split into several different business areas which are listed below:

\begin{itemize}
\item \Gls{gbs}
\item \Gls{gts}
\item \Gls{swg}
\item \Gls{stg}
\item Sales \& Distribution
\item \Gls{itd}
\item Integrated Managed Business Process Services
\item \Gls{ibm} Global Financing
\end{itemize}

I was employed under the \gls{it} scheme in \gls{swg}{United Kingdom}. Based at the Hursley site in 
Hampshire, the main \gls{swg} site in the \gls{uk}.

\Gls{swg}\footnote{\gls{ibm} is such a large company, it would take over 5,000 words to explain the
whole structure, so I shall focus on my specific working areas.} is split into five brands: DB2, 
Lotus, Tivoli, WebSphere and Rational. \Gls{swg} in Hursley is largely WebSphere-based. \\



\Gls{cics} Transaction Server\footnote{\Gls{cics} Transaction Server is typically shortened to CICS
internally} is a part of the WebSphere brand. Like many products in \gls{swg} \gls{cics} has 
several different departments: development, test (\gls{fv} test, system test, etc.) and service.

All service departments in \gls{ibm} are split into three distinct levels:

\begin{description}
\item[Level 1 Service] Are the first point of contact for customers. They deal with basic problems
with the product and have a general understanding of the product. If the problem can't be solved by
Level 1, it is elevated to Level 2. All problems reported to Level 1 are raised as a \gls{pmr} and
are tracked by \gls{retr}.
\item[Level 2 Service] Have a good working knowledge of the product and are typically able to 
diagnose and solve customer problems. If Level 2 are unable to diagnose the problem the \gls{pmr}
is elevated to Level 3 or, if the diagnosis reveals a problem with the product an \gls{apar} is
raised against Level 3. 
\item[Level 3 Service] Have a very good knowledge of diagnosing problems with the product and of
the internals of the product and are authorised to make changes to the source code of the product
to fix problems raised by \gls{apar}s. The majority of Level 3 work involves handling \gls{apar}s,
however some specialist members of the team handle \gls{pmr}s.
\end{description}

Due to the specialist knowledge required to work in the \gls{cics} Level 3 Service team I was not
expected to deal with either \gls{apar}s or \gls{pmr}s\footnote{It can take graduates up to a year
and a half to work without constant supervision.}. My main role in the team was to maintain 
existing tooling and to develop new tooling which would benefit the team. \\



At the start of the year my main responsibilities were to maintain a tool which would gather 
statistics on \gls{pmr}s and \gls{apar}s in \gls{retr} and a system named ``SPA'', a z/OS based
working environment specific to \gls{cics} Level 3 service. However, due to a process change about
a year before I joined \gls{ibm}, a new working environment; \gls{rtc} was being used for all new
releases of \gls{cics}.

I was initially tasked with integrating this environment into the 
statistics tooling, or vice versa as \gls{rtc} could potentially provide management and statistics
gathering by default. \\



After some changes to the team I was asked to change my focus to maintaining a large tool which
pulled \gls{apar}s from \gls{retr} into \gls{rtc}. This tool was also designed to perform other
functionality such as delivering fixes for the Eclipse-based suite of tools for \gls{cics} to
update sites and \gls{ibm}'s central fix management site - Fix Central.

Due to consistent issues and a lack of knowledge with this tooling, it was eventually decided to
switch from this tooling to \gls{apt} a tool built and maintained by the \gls{mq} Level 3 Service 
team and which was being considered being supported by the lab-wide build team. \\



This left a hole for the automation of delivering fixes for the suite of Eclipse-based tools 
\gls{cics} has. I was asked to develop a solution, \gls{edt}\footnote{Explorer is a shortening of 
\gls{cics} Explorer; the main Eclipse-based tool for \gls{cics}}, which had to be resilient to the
problems which had plagued the old tooling.

Developing this tooling also increased my exposure to the team's \gls{rtc} environment and lead to
me being partly responsible for maintaining the structure of work items (representations of 
\gls{apar}s and other associated tasks the team required to follow the service process). \\



Towards the end of my year I was also picked as part of a small team to plan and run the inductions
for the \gls{swg}-based \gls{it}s for 2012-13. As well as this I was a member of the main team for
Smart Cursor, a side project to continue the Extreme Blue project of the same name
(\href{http://www.bbc.co.uk/news/technology-14859157}{BBC News Article}).

\chapter{Technical and Application Environments}

Most systems I worked with in \gls{ibm} were either mainframe (z/OS or z/Linux systems), \gls{retr}
and SPA were both being z/OS application\footnote{Interesting point: \gls{retr} is a \gls{cics} 
application}; or Linux servers, typically \gls{rhel}.

Because \gls{cics} is almost entirely mainframe-based the department has access to several of the
on-site mainframes. However, with the introduction of \gls{rtc} the department has also required
the use of Linux servers to host the \gls{rtc} server on and to run build engines that hook into
\gls{rtc} and perform useful tasks, such as building \gls{cics} or polling \gls{apar}s into 
\gls{rtc} Work Items. Some of these were virtualised machines maintained by the Infrastructure team
in Hursley, whilst others were physical machines the department maintained. \\



The statistics tools I was responsible for were hosted on two different systems; the Java-based
tool ran on a \gls{rhel} CentOS 5 server running \gls{was}, \gls{mq} and \gls{db2}. Whilst the 
other was a Windows XP machine running Lotus Notes 8.5 and \gls{pcomm} sessions into \gls{retr} and
SPA. \\



For \gls{rtc} build engines, the department had access to several virtualised servers; one 
\gls{rhel} CentOS 4 server (winlnx0u.hursley.ibm.com) which was removed towards the end of my year
due to CentOS 4 going out of \gls{ibm} support and was replaced with two \gls{rhel} CentOS 6 
servers (cicspoller1.hursley.ibm.com and cicspoller2.hursley.ibm.com). All these servers ran 
\gls{rtc} 3.1.0.1 Build Engines which accessed the main departmental \gls{rtc} servers
(jazz104.hursley.ibm.com and jazz114.hursley.ibm.com). Another machine, local to the department
(hsm.hursley.ibm.com) was used as a back-up Build Engine machine, in case the server running the
virtualised servers should go down. HSM was also used as a testing environment for new releases of
\gls{rtc}.


\chapter{My role in IBM}

\section{Day Job}


\subsection{Inductions and Joining the CICS Level 3 Service Team}
When I first join I started with a two day induction in the North Harbour office in Portsmouth 
which covered the general environment I would be working, Health and Safety and the Business 
Conduct Guidelines. As well as some information about what I could do during the year. I received
my work laptop and met with my \gls{pdm}. \\



On the third day I started properly in the Hursley office and joined the \gls{cics} Level 3 Service
team. The team was formed of around 20-25 members including several Graduates and a another 
\gls{it} coming to the end of his year at \gls{ibm}

The current \gls{it}, Abul, assisted me with the set-up of my laptop and with settling into the 
role of maintaining the team's Java-based statistics tool which run on \gls{was}. This tool would 
poll \gls{retr} for \gls{pmr}s and \gls{apar}s on an hourly basis and also connected to SPA using 
\gls{mq}. This information was collected into a \gls{db2} database.

Every day a report would be generated for the previous day's receipts of \gls{pmr}s \gls{apar}s 
from the data in the database which was sent to the team leaders and the appropriate 
1\textsuperscript{st} and 2\textsuperscript{nd} Line Managers. It would also use the database to
create a similar report monthly, as well as performing a backup of data.

Finally, using \gls{j2ee} the tool provided a \gls{jsp} website which could be accessed by any 
member of the \gls{cics} Level 3 Service BlueGroup\footnote{Effectively an \gls{ldap} Group}.

I started out fixing some minor issues with this tool and learning the \gls{rtc} Client \gls{api}
in the effort to pull \gls{apar} data from the \gls{db2} database across to a view in the \gls{rtc}
eclipse client. \\



Soon after Abul left I was also given charge of a Lotus Notes-based statistics tool which collected
similar information as the aforementioned statistics tool. This one, however, was more manual and
required the maintainer to run a scan at least once per day. This tool directly interfaced with 
four \gls{pcomm} sessions, two of which connected to \gls{retr} and two to UKCPSG.

The \gls{retr} sessions handled \gls{pmr} and \gls{apar} polling. Collecting full information on
\gls{pmr} data, and basic data on \gls{apar} data. One of the UKCPSG sessions connected into SPA 
and gathered data on the current progress of \gls{apar}s, whilst the other ran an internal routine 
called QA which collated \gls{apar} information from different SPA sub-sections for a more full
report on the \gls{apar} in question. \\



\subsection{APAR Poller}

A few months into this I was asked to help maintain the team's \gls{apar} Poller; a large tool 
which polled \gls{apar}s in \gls{retr}, creating them as a work item in \gls{rtc} if the \gls{apar}
didn't already exist in the \gls{rtc} server or updating the existing work item with any changes
which had been made in \gls{retr}. This was originally designed to be a two-way bridge (changes in
the \gls{rtc} \gls{apar} work item could update the \gls{retr} \gls{apar}. However as permissions
for the \gls{rtc} work item were less secure as the ones in \gls{retr} so this bridge had been 
disabled).

The \gls{apar} Poller ran on a \gls{rtc} Build Engine; this build engine would connect into a 
\gls{rtc} server, which would hand the engine tasks to perform depending on build definitions on
the \gls{rtc} server. \\



My main role in maintaining this was to keep it working as some of the internal libraries it used
updated regularly and, contrary to guidelines, sometimes broke backwards-compatibility. This 
proved to be difficult with the sheer size of the application and the way in which it was invoked.

The application was invoked using Apache Ant; an open source tool similar to GNU Make, but designed
for Java applications. Whilst this in itself was a good design decision, the readability of the 
Ant scripts was less than desirable. To add to this some of the variables were taken from the 
Build Engine Definition on the \gls{rtc} server, some from the Build, some from flat files and some
hard coded into the Ant scripts. This made it nearly impossible to debug.

Mark Richards, a fellow member of the \gls{cics} Level 3 Service team also charged with maintaining
the poller, and I both raised our concerns for the maintainability of the Poller in the long run.
With these concerns aboard a number of meetings were set up with the \gls{mq} Level 3 Service team
to discuss their solution, \gls{apt}.

\gls{apt} performed very similar function as the \gls{apar} Poller did; however it was a lot more
simplistic. Despite this it was being considered by the Hursley Build Team as a lab-wide solution
which they could maintain easily. Due to this the \gls{cics} Level 3 Service team had a vested
interest in this solution and, after a fair few meetings, came up with a list of requirements we
would need to be added to \gls{apt} for us to replace the poller with. \\


\subsection{CICS Trace Entries}

During this time I was also given another piece of work which regarded the documentation of trace
messages in one of the \gls{cics} domains. I was set-up with read-only access to the \gls{cics} 
3.1, 3.2 and 4.1 source code on SPA and tasked with checking the difference between the source code
definitions of trace codes and the documented versions on the \gls{ibm} information centre to check
for inconsistencies (i.e. added or removed entries and entires with incorrect parameters) and note
down the correct version to be sent to UTD, the department which handles documentation. \\

\subsection{UHL Printer Plug-in}

Almost immediately after completing this work I was asked to create a small eclipse plug-in to print
the team's \gls{uhl} \gls{rtc} work items in a similar form to the old version done by SPA. The 
\gls{uhl} is a report of all the work which goes into diagnosing and fixing an \gls{apar} and are
used heavily in the \gls{apar} reviews. They are also used if an \gls{apar} goes \gls{pe}, where
the fix causes a problem, to identify why the \gls{apar} went \gls{pe} and how the process can be
changed to prevent such action happening again.

This UHL Printer Plug-in used the \gls{swt} Printer libraries, along with the \gls{rtc} 
Plain-client libraries, to gather information from the currently selected \gls{uhl} \gls{rtc} work
item, build it into a document tree and finally parse this tree and print it straight to the 
printer. From this I then generated an Update-Site and associated elements and hosted it on one of
the departmental servers for the team to easily access. \\


\subsection{Explorer Delivery Tool}

With this came another problem. In the \gls{cics} Explorer and associated Eclipse-based tools 
branch of the poller code was a mechanism to deliver \gls{apar} fixes to the \gls{ibm} 
update-sites and to Fix Central; the centralised location for the majority of fix packs. Due to the
nature of \gls{cics} this site was only used for the eclipse-based tools\footnote{A fix for 
\gls{cics} are released as a \gls{ptf}; a common mechanism for z/OS applications.}.

This functionality was a useful step for the few members of the team who worked with the 
Eclipse-based tools in the delivery of fixes. Mark Richards was the one to request that I re-write
a more maintainable version of this code using \gls{osgi} and following good practices for Java,
\gls{osgi} and the IBM guidelines. To start with I had a lot of discussions with Mark about the 
best structure for the application. From there we decided \gls{osgi} was the best way forward due 
to it's focus on modularity; if one section broke due to updating libraries it would remain 
isolated to that section of the application without affecting any other features.

With this decision in place we then began coming up with a decent structure for the application
which would leverage \gls{osgi}. We also discussed how we would handle auditing and object 
serialisation in a readable format. To this end we decided to use Jackson, an Open Source library
under the Apache License produced by Codehaus, which serialised Java Objects into \gls{json} files
through the use of Java annotations. \\

\subsubsection{EDT Document Generation}

The first part of the process was to generate documentation for the fix to be delivered. To do this
we turned to another Open Source library; Apache Velocity. This library acts as a templating engine
allowing Java values to be passed into a predefined template with relative ease. Again this is
licensed under the Apache License.

To do this a set of \gls{pojo}s were also defined, these would be the objects which Jackson would
serialises and de-serialise to \gls{json} files. \\



With this complete I demonstrated the workings to both my Task Manager and Team Leader. With their
approval I continued the project and ended up with weekly progress meetings with them. \\

\subsubsection{EDT RTC Interaction}

The next part was adding functionality to store files to a \gls{scm} system. This would allow 
auditing on all items in the application. Because of the heavy reliance on \gls{rtc} in the whole
department and due to some issues with connectors to other \gls{scm} systems, it was decided to use
\gls{rtc} as the \gls{scm} system.

At this time I also decided to write the connector to the work item side of \gls{rtc}. This worked
slightly different from the code in the \gls{uhl} Printer so I chose to write it from scratch to
make it cleaner and more maintainable. This work item module read certain types of work item on the
server and would populate the correct \gls{pojo} with information from this work item. \\

\subsubsection{EDT Fix Central Interaction}

Up until this point I hadn't had any problems integrating libraries into the application, both
Jackson and Velocity were contained in single jar files. The \gls{rtc} libraries were large, but
due to the Eclipse-based nature of \gls{rtc} they were already designed to be integrated into 
\gls{osgi}.

However, when it came to integrate Fix Central's libraries into \gls{edt}, I started running into
dependency problems. The problem had occurred as the Fix Central Library had dependencies on
certain external libraries which contained different versions of fairly common libraries. Due to
the way \gls{osgi} performs class loading it resolved these dependencies in a different order to
the way standard Java would.

With Mark's help I was eventually able to resolve this and slowly learn the Fix Central library and
finally use it to create the required metadata and upload fixes to a testing server. A further part
of this was the ability to upload in preview mode and then to unset the preview flag. This took a
lot more time due to the convoluted way Fix Central works. \\

During this I was constantly testing making sure each module would integrate correctly with other
modules. This combined with complex libraries meant the whole project took a lot long than first
anticipated. Eventually I started to do some proper project planning on the request of my team 
leader. \\

\subsection{CICS Eclipse-based Tools Skills Transfer}

Around the release of new versions of the \gls{cics} Eclipse-based tools I was asked to moderate
and record the skills transfer sessions for these new versions. These skills transfer sessions were
designed to bring Service teams, both Level 2 and 3, up to speed on the new features of the product
and how they are designed to work.

Most of these sessions were delivered by foreign teams so my role could be difficult at time
working with due to the language barrier. However all these sessions ran fairly smoothly, where
the only problems came from the software we were using to record. \\

\section{Giveback}

\gls{ibm} has a scheme known as ``giveback'' which provides opportunities to help with community
events both within \gls{ibm} and externally. All \gls{it}s are pushed by their managers to 
participate in giveback.

\subsection{Java Teaching}

The first piece of giveback I participated in was teaching Java to my peers. These gls{it}s had a
good knowledge of the C programming language so the main focus for this was teaching the principals
of object-orientation and then some of the differences in syntax. To help with this we were given a
mentor who had a good knowledge of teaching and lecturing.

This piece of giveback lasted around 3 months with a session a week and by the end of it the 
\gls{it}s in the group knew Java well enough to use it in their own day jobs.

Having heard about these sessions some of my team registered their own interest in learning Java
and as such I set up a couple of sessions to teach them too. \\

\subsection{Smart Cursor}

Having followed Extreme Blue (a summer internship program \gls{ibm} runs world-wide) with interest
due to some friends being in that year's intake I jumped at the chance to get involve with the 
continuation of one of these projects; the Smart Mouse project.

The Smart Mouse project is a hardware and software project designed to allow disabled users to map
mouse movements to gestures from a 9-degrees of freedom sensor they could wear on any part of their
body.

The prototype the Extreme Blue students had produced in their 8 week internship was very simplistic
and would only work with certain gestures. Our aim was to allow users to use any gesture they 
wanted (after training the system) along with the ability to attach multiple sensors or other
peripherals such as pupil tracking or microphones.

I joined what was known as the ``A-Team'' (Algorithm Team) and got involved with designing and 
building the base for this project, including a 3D visualisation program which would be used to 
debug the actual \gls{ai} sections of the project. \\

\subsection{Blue Fusion}

Another large piece of giveback both graduates and \gls{it}s are pushed to do is an event called
``Blue Fusion''; an event which coincides with national science week, with the aim to teach school
children concepts of science and technology. I joined a team of 6 to create one of the games for
this event which was also going to be used for the grand finale.

Most of my work on this game was the 2D graphics for the front-end of the game. I also spent a lot
of time testing the performance of the game to ensure it ran well enough for the low-end hardware
we would be given for the event. I also helped to write some of the \gls{ai} behaviour for the 
game.

During the project I acted as the team's 2\textsuperscript{nd} in command, standing in for the team
leader in the weekly check-up meetings and keeping the team on task while he was away on vacation.
On request of my task manager I also gave a quick demo of the game to the rest of the \gls{cics}
level 3 team, which was a useful experience to get opinions on the game from an impartial audience.
\\



As a developer I had to host a session for the ``Bright Sparks'' event; a single day ran just
before Blue Fusion for a younger audience which acts as a final test where most problems can be
caught and fixed. I spent the morning of Bright Sparks hosting the game and the afternoon 
performing some simple bug fixing in preparation for the finale. All-in-all the day was a little
ropey, but all things considered ran well.

During Blue Fusion I spent a lot of time doing minor bug fixes and making sure all the data files 
were correct for the finale. I also hosted one of the other games for a day due to a shortage of
volunteers. For the majority of the event our game was stable and the only issues occurred due to
some problems with the Java 2D libraries. \\



\subsection{Java Masterclass at Swanmore College of Technology}

Having done a lot of Java teaching in the first part of my year, I responded to an advertisement
looking for helpers to run Java teaching sessions in a local school. This led to me both running
and organising these sessions. Contacting the teacher who'd requested these sessions we organised
a taster session just before the Easter holidays to work out the levels of interest so the main
sessions could begin after the Easter break.

All the sessions were run in a similar format; a quick presentation for 5-15 minutes followed by a
practical worksheet the students could work through with any help from me or one of the other 
helpers.

For these sessions I spent a fair amount of time beforehand trying to find helpers for the sessions
and checking through the content for the session to make sure the practical sessions would work 
well or shortening the presentations. \\



\subsection{2012-13 IT Inductions}

Towards the end of my employment at \gls{ibm} I was asked to organise the \gls{it} inductions for
the 2012-13 intake of \gls{it}s. The team for this consisted of myself and two other \gls{it}s, 
supervised by Sarah Churchard, one of the three Hursley \gls{it} managers; and Ed Moffat, the lead
manager for the Hursley \gls{it}s.

We quickly formulated a timetable for the typical 2 day induction and started finding and booking
speakers to run some of the sessions we had planned. These speakers included: the Hursley 
Laboratory Manager, two Senior Inventors (\gls{ibm}ers with several patents filed), several of the 
2011-12 \gls{it}s and some graduates from different product areas.

As well as this we also needed to book rooms and food for the event. Alongside this we requested
laptops for each new intake which we then set-up to make the first few days easier for the 
\gls{it}s. 

Being one of the more technical of the three, I was responsible for this set-up. With some help I
added Hard-disk passwords to every laptop; changed the Windows and Lotus Notes passwords to a 
default so the owner could easily set their own; changed to the regional options to be the UK
locale; enabled the replication settings on Lotus Notes; created an internal ID for the owner and
linked it to a wireless account.

During the day I was on hand to greet and introduce speakers, answer any questions, guide the 
\gls{it}s around the site and fix any laptop problems which arose during the set-up sessions.
The first two inductions ran very successfully and due to limited numbers of people in the final
induction I gave a half-day induction to a single new \gls{it}, which again ran smoothly.

Due to our efforts on the induction we were thanked by Derek Tregus; the aforementioned Hursley 
Lab Manager and by Ed and Sarah. \\



\chapter{Critical Evaluation}

Having been invited to several reviews as part of the Level 3 \gls{cics} Service team I was 
surprised at the level of detail put into each review no matter how many lines of code had been
changed. For example: one of the reviews I attended was for two trivial lines of code and yet every
single possibility was run through; how it might affect any other modules, what would happen with
other versions, etc. It was even decided that development should also approve the fix before it was
released due to some unknowns.

Therefore, it was a little surprising that a lot of the side-products for \gls{cics} didn't undergo
the same standards in their reviews (possibly due to the fact that they couldn't go \gls{pe}). It
also seemed a lot of these products were coded without following the \gls{ibm} coding guidelines or
the guidelines for the language they were written in. It also seemed like there was a dangerous
misconception over the use of Open Source Software, despite there being lawyers to ensure 
otherwise. \\

One thing that really impressed me with \gls{ibm} was that there was never any prejudice based on
my position as an \gls{it}. The only thing which affected my work was the limited amount of time I
had in \gls{ibm}. Most of the time fellow employees would mistake me for a graduate, even those 
inside my own team. \\

Technically I learned a lot over the course of the year. As I have maintained to some of the new
\gls{it}s; I came in with 2 years knowledge of Java and came out with a single year of solid use 
of Java. I also spent a lot of time learning the \gls{osgi} framework and have come to appreciate
being able to produce highly-modular Java applications when necessary. \\

Having worked with mainframes for a year I have definitely learnt a lot about z/OS and associated
operating systems and plan to take the z/OS Mastery Qualification, as well as a Java 7 
qualification too. \\

One of the most important skills I have improved over the course of this year is the ability to
learn an external library quickly despite the state of it's documentation. Hand-in-hand with this
I have also significantly improved my ability to debug programs, especially with the in-built 
eclipse graphics debugger. \\

Presenting is a big part of working in \gls{ibm}. Over the year I have ended up doing a lot of 
presenting, from teaching to explaining the technical workings of my projects. I also took the
opportunity to attend a presentation workshop with a member of the executive briefing centre and
feel a lot more confident giving presentations. \\

Having spent a lot of time in teams I feel my teamwork has improved and whilst I still don't feel
entirely confident leading a team, I am more willing to throw in more opinions regardless of my
place within the team. \\

I also had some great learning opportunities to learn a great deal more about \gls{cics} and other
\gls{ibm} products, as well as interesting topics like virtualisation and machine learning. As well
as being able to do some of my own teaching and presenting. I actually ended up doing part of a
\gls{cics} seminar along with some of the other \gls{cics} \gls{it}s. \\

All-in-all this year at \gls{ibm} has boosted my confidence, both with technology and 
communication. I also have a much greater appreciation for what it's like to work in a large 
software (and hardware) company.
\end{document}
