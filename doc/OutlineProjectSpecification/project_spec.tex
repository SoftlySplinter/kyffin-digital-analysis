\documentclass[11pt,fleqn,twoside]{article}
\usepackage{makeidx}
\makeindex
\usepackage{palatino} %or {times} etc
\usepackage{plain} %bibliography style 
\usepackage{amsmath} %math fonts - just in case
\usepackage{amsfonts} %math fonts
\usepackage{amssymb} %math fonts
\usepackage{lastpage} %for footer page numbers
\usepackage{fancyhdr} %header and footer package
\usepackage{mmp} 
\usepackage{url}
\usepackage{cite}
\usepackage{glossaries}

\newacronym{opencv}{OpenCV}{Open Source Computer Vision}
\newacronym{fiji}{FIJI}{FIJI Is Just ImageJ}
\newacronym{vxl}{VXL}{Vision X Library}
\newacronym{ivt}{IVT}{Intergrating Vision Toolkit}
\newacronym{sip}{SIP}{Scillab Image Processing}

\begin{document}

\name{Alexander D. Brown}
\userid{adb9}
\projecttitle{Kyffin Williams: Digital Image Analysis}
\projecttitlememoir{Kyffin Williams: Digital Image Analysis} %same as the project title or abridged version for page header
\reporttitle{Outline Project Specification}
\version{0.1}
\docstatus{Draft}
\modulecode{CS39440}
\supervisor{Hannah M. Dee} % e.g. Neil Taylor
\supervisorid{hmd1}

%optional - comment out next line to use current date for the document
%\documentdate{22nd April 2012} 
\mmp

\setcounter{tocdepth}{3} %set required number of level in table of contents


%==============================================================================
\section{Project description}
%==============================================================================
%Two or three reasonably detailed paragraphs describing what the project is about. Make clear the main substance of the project, those aspects of it that are essential in making it worthwhile, and the end-goals of the project. You might provide some citations.
Sir John ``Kyffin'' Williams was a Welsh Landscape Painter, widely regarded as the defining artist of Wales during the 20\textsuperscript{th} Century\cite[p.957-958]{davies08}. His work, and associated metadata collected by the National Library of Wales, allows for some interesting analysis; particularly that of temporal or geological %FIXME is this the right word?
data for a given painting.

Temporal analysis will be the focus of this project as it allows for a diverse range of techniques; from statistical analysis of RGB values of the paintings to looking at the length and style of paintbrush strokes. The ultimate aim of this being to accurately place the year of a given painting which has no metadata collected.

%==============================================================================
\section{Work to be tackled}
%==============================================================================
%This should summarise the tasks that will form the major part of the work. If there is lots of research reading to be done you can say so here. If there are coding techniques to learn or APIs to wrestle with then you can say that too.

\subsection{Library and Language Decisions}
There are quite a few image processing/computer vision libraries available to use, including:

\begin{itemize}
\item AForge.net
\item \gls{fiji}
\item \gls{ivt}
\item \gls{opencv}
\item \gls{sip}
\item \gls{vxl}
\end{itemize}

Some analysis of these libraries needs to be performed before a proper choice is made. This analysis should take into account certain aspects such as cross-platform compatibility, ease of use and install, features available, etc.



%The \gls{opencv} library contains a lot of useful methods for image processing and therefore makes it an important library to research and, probably, include as part of this project. It has C, C++, Python and Java interfaces so some work will also need to be done researching the use of it with each of these languages to decide which would be the best to use with this project. 

%Currently I have a very good working knowledge of Java and a basic knowledge of C, C++ and Python. However other maths- or science-based libraries; such as NumPy, \gls{gsl} or JAMA; may also affect the decision made.


%==============================================================================
\section{Project deliverables}
%==============================================================================
%This should list everything you expect to produce. This should normally include specified items of working software, any reviews (of technology etc...) that you see as of fundamental importance to the project, documentation for requirements, design and testing, and of course the progress report and the final report. 

%This list should include more than just the items that have been noted in the lecture slides. You must focus on the items that you will produce for your project and it might be different for your project than for another project. The list might change as you continue to work on your project. This is a statement of your current expectations.  

%==============================================================================
\section{Initial bibliography}
%==============================================================================
%This should list things that you expect to read to help with your project. For example, your supervisor may provide you with some references that give essential background to your project and you are likely to need to do some preliminary reading of the scientific literature, text books, and internet material.

%The purpose of the section is to understand what sources you are looking at and expect to be looking at.  For this document, a list of items is sufficient. You might go further and make use of bibliographic tools, e.g. BibTeX in a LaTeX document, could be used to provide citations, for example \cite{NumericalRecipes}\cite{MarksPaper}\cite{FailBlog}\cite{kittenpic_ref}.  The bibliographic tools are not a requirement in this document, but you are welcome to use them.  

% example of including
\bibliographystyle{plain}
\renewcommand{\refname}{}  % if you put text into the final {} on this line, you will get an extra title, e.g. References. This isn't necessary for the outline project specification. 
\bibliography{mmp} % References file


\end{document}
