\documentclass[11pt,fleqn,twoside]{article}
\usepackage{makeidx}
\makeindex
\usepackage{palatino} %or {times} etc
\usepackage{plain} %bibliography style 
\usepackage{amsmath} %math fonts - just in case
\usepackage{amsfonts} %math fonts
\usepackage{amssymb} %math fonts
\usepackage{lastpage} %for footer page numbers
\usepackage{fancyhdr} %header and footer package
\usepackage{mmp} 
\usepackage{url}
\usepackage{cite}
\usepackage{glossaries}

\newacronym{opencv}{OpenCV}{Open Source Computer Vision}
\newacronym{fiji}{FIJI}{FIJI Is Just ImageJ}
\newacronym{vxl}{VXL}{Vision X Library}
\newacronym{ivt}{IVT}{Intergrating Vision Toolkit}
\newacronym{sip}{SIP}{Scillab Image Processing}

\begin{document}

\name{Alexander D. Brown}
\userid{adb9}
\projecttitle{Kyffin Williams: Digital Image Analysis}
\projecttitlememoir{Kyffin Williams: Digital Image Analysis} %same as the project title or abridged version for page header
\reporttitle{Outline Project Specification}
\version{1.0}
\docstatus{Final}
\modulecode{CS39440}
\supervisor{Hannah M. Dee} % e.g. Neil Taylor
\supervisorid{hmd1}

%optional - comment out next line to use current date for the document
%\documentdate{22nd April 2012} 
\mmp

\setcounter{tocdepth}{3} %set required number of level in table of contents


%==============================================================================
\section{Project description}
%==============================================================================
%Two or three reasonably detailed paragraphs describing what the project is about. Make clear the main substance of the project, those aspects of it that are essential in making it worthwhile, and the end-goals of the project. You might provide some citations.
Sir John ``Kyffin'' Williams was a landscape painter from Wales who's work was predominantly based in Wales and Patagonia. His work, and associated metadata collected by the National Library of Wales, allows for some interesting analysis; particularly that of temporal or geological %FIXME is this the right word?
data for a given painting.

Temporal analysis will be the focus of this project as it allows for a diverse range of techniques; from statistical analysis of RGB values of the paintings to looking at the length and style of paintbrush strokes. The ultimate aim of this being to accurately place the year of a given painting which has no metadata collected.

%==============================================================================
\section{Work to be tackled}
%==============================================================================
%This should summarise the tasks that will form the major part of the work. If there is lots of research reading to be done you can say so here. If there are coding techniques to learn or APIs to wrestle with then you can say that too.

\subsection{Library and Language Decisions}
There are quite a few image processing/computer vision libraries available to use, including:

\begin{itemize}
\item AForge.net
\item \gls{fiji}
\item \gls{ivt}
\item \gls{opencv}
\item \gls{sip}
\item \gls{vxl}
\end{itemize}

Some analysis of these libraries needs to be performed before a proper choice is made. This analysis should take into account certain aspects such as cross-platform compatibility, ease of use and install, features available, etc.


\subsection{Image Processing and Computer Vision Research}
Having never done any digital image processing or computer vision it will be vital to research into the area. For some of the simpler techniques; RGB Statistical Analysis, for example, this is less of an issue. 

For the long term viability of the project this needs to be completed before any major work.

I intend to use Hannah and other staff members of the Computer Science Department to aid with this as there is a wealth of knowledge existing there.


\subsection{Base System}
As with any software project, a strong base will allow for a good overall whole. The base for this system will need to allow the running of different analysis techniques on lists of images and for the output of this to be passed into a machine learning module in a form which allows for the classification of a single image using the given technique.

The difficulty of this will be passing the techniques around in this fashion. C allows method pointers, any Object Orientated language can do this with help from inheritance, etc.

\subsection{Techniques}
The main focus of this project is on the Techniques to analyse the paintings. So creating these techniques will have to be a large piece of work.

Part of this will be coming up with the ideas for these techniques; there are currently a few ideas for these techniques:

\begin{itemize}
\item RGB Statistical Analysis (Mean, Standard Deviation, Range, etc.)
\item Histogram Comparison
\item Texture Analysis
\item Brush stroke Analysis
\end{itemize}

The more of these techniques there are the better the analysis can be and the more likely the project is to be able to accurately predict the age of a painting.


\subsection{Machine Learning}
The final required part of this is to be able to ``Learn'' the age of paintings given the analysis performed upon it, map a function which matches all learned information from this information and be able to infer the age of a new painting.

This part is potentially the most difficult as such systems can be fairly difficult to create. However there may be some libraries which include Machine Learning packages. I will need to explore these libraries and find out if they are suitable for use.


%==============================================================================
\section{Project deliverables}
%==============================================================================
%This should list everything you expect to produce. This should normally include specified items of working software, any reviews (of technology etc...) that you see as of fundamental importance to the project, documentation for requirements, design and testing, and of course the progress report and the final report. 

%This list should include more than just the items that have been noted in the lecture slides. You must focus on the items that you will produce for your project and it might be different for your project than for another project. The list might change as you continue to work on your project. This is a statement of your current expectations.  

\begin{enumerate}
\item Requirements Documentation.
\item Review of Computer Vision Libraries.
\item Review of Programming Language Choice (where applicable).
\item Research into Image Processing and Computer Vision.
\item Design Documentation.
\item Implement system skeleton.
\item Test Documentation
\item Implement unit and system tests.
\item Complete system implementation.
\item Implement Machine Learning module.
\item Design and implement a number of Painting Analysis Techniques.
\item Progress Report.
\item Final Report. 
\end{enumerate}

%\subsection{Analysis of Computer Vision Libraries}
%The first step of this project is to decide on a suitable Computer Vision library. To do this a comparison table will need to be created and completed for all applicable libraries. Alongside this, where there is a choice, analysis of potential languages that can be used with the chosen library will need to be completed too.

%==============================================================================
\section{Initial bibliography}
%==============================================================================
%This should list things that you expect to read to help with your project. For example, your supervisor may provide you with some references that give essential background to your project and you are likely to need to do some preliminary reading of the scientific literature, text books, and internet material.

%The purpose of the section is to understand what sources you are looking at and expect to be looking at.  For this document, a list of items is sufficient. You might go further and make use of bibliographic tools, e.g. BibTeX in a LaTeX document, could be used to provide citations, for example \cite{NumericalRecipes}\cite{MarksPaper}\cite{FailBlog}\cite{kittenpic_ref}.  The bibliographic tools are not a requirement in this document, but you are welcome to use them.  

% example of including
\bibliographystyle{plain}
\renewcommand{\refname}{}  % if you put text into the final {} on this line, you will get an extra title, e.g. References. This isn't necessary for the outline project specification. 
\bibliography{mmp} % References file


\end{document}
