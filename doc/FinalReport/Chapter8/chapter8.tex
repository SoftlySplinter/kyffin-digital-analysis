\section{Conclusions}


To the best of the authors knowledge this is the first work that attempts to date work by an artist by 
year. Similarly, it is believed that this is the first attempt to try and perform digital analysis
of paintings from a range of catalogue and web images.

The results presented here show that computer vision \emph{can} help with the job of dating art 
within an artist's body of work. It has been shown that strong, statistically significant 
correlations between a method's allocation of year and the actual year of painting; as well as the
ability to classify 70\% of paintings to within their actual year of painting (within a dataset 
that spans 6 decades). These results are not yet of great use to art historians, but it is hoped 
that future work will be able to improve upon this. Several statistical avenues remain to be 
explored: looking into feature combination and selection, and also investigate the potential for 
treating the year classification problem not as a nearest neighbour problem, but as an ordinal 
regression problem. 

Future directions will also involve testing the methods presented here on the works of other 
artists who have shown great stylistic variation over the course of their career: one plan was to 
build a dataset of, for example, David Hockney works.  Whilst this test has not yet been performed
it is hoped it would be a success: by avoiding brushstroke detection (which we expect to be artist
specific) we hope to have developed techniques with application across a broader range of artistic
styles, and by building techniques which work on catalogue images rather than those captured in 
controlled conditions, there is a better openness to working with paintings from a wider range of 
artists.


There are several larger projects that could be considered for continuation from this project. 
Being able to display a graphical representation of the analysis performed, for each painting in a
publicly-accessible format. This would take the form of a website with HTML5 elements and would be used as an aid for teaching digital 
image processing.

Another venture was to create a 3D model of Williams' work; due to the highly textural nature of
his work this could provide more interesting analysis techniques. This is also a novel way of being
able to showcase an artist's work for those who do not have access to the physical work. Again
this would ideally use HTML5 elements to make the work available to the general public.

As this project never considered brush-stroke analysis it would be interesting to see how 
generalised some of the brush-stroke detection algorithms mentioned in 
Section~\ref{sec:existing-brush-stroke} would work on an artist who primarily used a palette knife. Perhaps more interestingly, how these techniques could be modified to apply them too the palette 
knife strokes. A further work from this would be the ability to differentiate between different
painting implements.


Unfortunately, there is a sticking point when it comes to this project. Using painting size to 
classify the year of a painting produces the best correlation of all other analysis techniques.
There may be a mundane reason for this: cost.

At the beginning of his career Williams would have only been able to afford small canvases, 
especially with the amount of paint he used. As his career progressed he would have been able
to afford increasingly large canvases and the paint associated. However, as this project is wanted
to be generalised to other artists this may not hold true for all catalogues of differing artists.

However, as a paper that is based on the work detailed in this project has been submitted to the
8th International Symposium on Image and Signal Processing. The author feels reasonably well 
assured in saying that this project was a great success.

