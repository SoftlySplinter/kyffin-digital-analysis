\chapter{Code Samples}

\section{Example CSV Parsing Code}
\begin{lstlisting}[language=python,breaklines=true,
caption={Example CSV Parsing Code from http://docs.python.org/2/library/csv.html},label=lst:csv_example_code]
>>> import csv
>>> with open('eggs.csv', 'rb') as csvfile:
...     spamreader = csv.reader(csvfile, delimiter=' ', quotechar='|')
...     for row in spamreader:
...         print ', '.join(row)
Spam, Spam, Spam, Spam, Spam, Baked Beans
Spam, Lovely Spam, Wonderful Spam
\end{lstlisting}

\section{argparse Example Code}
\begin{lstlisting}[language=python,
caption={Example argparse Code from http://docs.python.org/2/library/argparse.html},
label=lst:argparse_example_code]
import argparse

parser = argparse.ArgumentParser(description='Process some integers.')
parser.add_argument('integers',
                   metavar='N',
                   type=int, 
                   nargs='+',
                   help='an integer for the accumulator')
parser.add_argument('--sum',
                   dest='accumulate',
                   action='store_const',
                   const=sum, default=max,
                   help='sum the integers (default: find the max)')

args = parser.parse_args()
print args.accumulate(args.integers)
\end{lstlisting}

\section{Gabor Filter Example Implementation}
\begin{lstlisting}[language=MATLAB,breaklines=true,
caption={Example implementation of a Gabor Filter in MATLAB from wikipedia\cite{Contributors2012Gabor}}]
function gb=gabor_fn(sigma,theta,lambda,psi,gamma)
 
sigma_x = sigma;
sigma_y = sigma/gamma;
 
% Bounding box
nstds = 3;
xmax = max(abs(nstds*sigma_x*cos(theta)),abs(nstds*sigma_y*sin(theta)));
xmax = ceil(max(1,xmax));
ymax = max(abs(nstds*sigma_x*sin(theta)),abs(nstds*sigma_y*cos(theta)));
ymax = ceil(max(1,ymax));
xmin = -xmax; ymin = -ymax;
[x,y] = meshgrid(xmin:xmax,ymin:ymax);
 
% Rotation 
x_theta=x*cos(theta)+y*sin(theta);
y_theta=-x*sin(theta)+y*cos(theta);
 
gb= exp(-.5*(x_theta.^2/sigma_x^2+y_theta.^2/sigma_y^2)).*cos(2*pi/lambda*x_theta+psi);
\end{lstlisting}


\section{OpenCV Histogram Example Code}


\begin{lstlisting}[language=python,breaklines=true,label=lst:opencv_histogram,
caption={Example Histogram calculation and displaying code from OpenCV\cite{opencv_library}.}] 
# Taken from: http://opencv.willowgarage.com/documentation/python/imgproc_histograms.html#calchist
# Calculating and displaying 2D Hue-Saturation histogram of a color image

import sys
import cv

def hs_histogram(src):
    # Convert to HSV
    hsv = cv.CreateImage(cv.GetSize(src), 8, 3)
    cv.CvtColor(src, hsv, cv.CV_BGR2HSV)

    # Extract the H and S planes
    h_plane = cv.CreateMat(src.rows, src.cols, cv.CV_8UC1)
    s_plane = cv.CreateMat(src.rows, src.cols, cv.CV_8UC1)
    cv.Split(hsv, h_plane, s_plane, None, None)
    planes = [h_plane, s_plane]

    h_bins = 30
    s_bins = 32
    hist_size = [h_bins, s_bins]
    # hue varies from 0 (~0 deg red) to 180 (~360 deg red again */
    h_ranges = [0, 180]
    # saturation varies from 0 (black-gray-white) to
    # 255 (pure spectrum color)
    s_ranges = [0, 255]
    ranges = [h_ranges, s_ranges]
    scale = 10
    hist = cv.CreateHist([h_bins, s_bins], cv.CV_HIST_ARRAY, ranges, 1)
    cv.CalcHist([cv.GetImage(i) for i in planes], hist)
    (_, max_value, _, _) = cv.GetMinMaxHistValue(hist)

    hist_img = cv.CreateImage((h_bins*scale, s_bins*scale), 8, 3)

    for h in range(h_bins):
        for s in range(s_bins):
            bin_val = cv.QueryHistValue_2D(hist, h, s)
            intensity = cv.Round(bin_val * 255 / max_value)
            cv.Rectangle(hist_img,
                         (h*scale, s*scale),
                         ((h+1)*scale - 1, (s+1)*scale - 1),
                         cv.RGB(intensity, intensity, intensity), 
                         cv.CV_FILLED)
    return hist_img

if __name__ == '__main__':
    src = cv.LoadImageM(sys.argv[1])
    cv.NamedWindow("Source", 1)
    cv.ShowImage("Source", src)

    cv.NamedWindow("H-S Histogram", 1)
    cv.ShowImage("H-S Histogram", hs_histogram(src))

    cv.WaitKey(0)]
\end{lstlisting}


