\chapter{Evaluation}

%Examiners expect to find in your dissertation a section addressing such questions as:

%\begin{itemize}
%   \item Were the requirements correctly identified? 
%   \item Were the design decisions correct?
%   \item Could a more suitable set of tools have been chosen?
%   \item How well did the software meet the needs of those who were expecting to use it?
%   \item How well were any other project aims achieved?
%   \item If you were starting again, what would you do differently?
%\end{itemize}

%Such material is regarded as the most important part of the dissertation; it should demonstrate that you are capable not only of carrying out a piece of work but also of thinking critically about how you did it and how you might have done it better. This is seen as an important part of an honours degree. You are expected to realise in which ways it falls short of perfection and of things that you did wrong.

%Sadly, the critical evaluation is the weakest aspect of most project dissertations. Because of its importance, some examples are provided on the project website.

\section{Evaluation of Requirements}


\section{Evaluation of Design}


\section{Evaluation of Tools}

\subsection{Programming Language}
I felt Python was a good choice of programming language; it's dynamically typed nature allows a
lot less restrictions and though on the initial design, fitting well with my choice of 
methodology.

Python was also very useful for built-in features like list comprehension (see 
listing~\ref{lst:python-list-comp} for an example) and other operations on lists such as 
\texttt{zip} on two arrays to help graph results.

\begin{lstlisting}[language=python,
caption={Example of List Comprehension in Python},
label=lst:python-list-comp]
year = [painting.year for painting in self.paintings 
        if is_absolute(painting.year)]
\end{lstlisting}

Python also boasts a good number of libraries by default; libraries like the \gls{csv} parsing 
library used to read in data.

\subsubsection{Dependency Management}


\subsection{Image Processing/Computer Vision Libraries}


\subsection{Machine Learning Libraries}

\subsubsection{File Formats}


\subsection{Scientific and Numeric Libraries}


