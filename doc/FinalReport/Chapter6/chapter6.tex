\chapter{Evaluation}

%Examiners expect to find in your dissertation a section addressing such questions as:

%\begin{itemize}
%   \item Were the requirements correctly identified? 
%   \item Were the design decisions correct?
%   \item Could a more suitable set of tools have been chosen?
%   \item How well did the software meet the needs of those who were expecting to use it?
%   \item How well were any other project aims achieved?
%   \item If you were starting again, what would you do differently?
%\end{itemize}

%Such material is regarded as the most important part of the dissertation; it should demonstrate that you are capable not only of carrying out a piece of work but also of thinking critically about how you did it and how you might have done it better. This is seen as an important part of an honours degree. You are expected to realise in which ways it falls short of perfection and of things that you did wrong.

%Sadly, the critical evaluation is the weakest aspect of most project dissertations. Because of its importance, some examples are provided on the project website.

\section{Evaluation of Requirements}
Due to this being a research project, the requirements were not well defined other than the end 
result of being able to classify paintings. Other requirements were added during meetings as new
problem areas were encountered but were never formally defined, this would have been a problem if 
the methodology used wasn't able to keep up with changing requirements.

Most of the aims that were defined to begin with were achieved; in fact all but brush-stroke 
analysis were completed effectively. However, on further investigation the completion of 
brush-stroke analysis was infeasible for this project, the area is rather large and complex and, 
with various time constraints, turned out that exemplars were a more revealing, less 
time-consuming area of investigation.


\section{Evaluation of Design}


\section{Evaluation of Tools}

\subsection{Programming Language}
Python is a dynamic programming language which offers clear, reliable syntax; object orientation
including multiple inheritance; full modularity; exception-based error handling; and a good range
of dynamic data types.

Python was a good choice of programming language; it's dynamically typed nature allows a lot fewer
restrictions and though on the initial design, fitting well with the choice of methodology. 

Python was also very useful for built-in features like list comprehension (see 
listing~\ref{lst:python-list-comp} for an example), keeping the amount of code needed to build 
list- and dictionary- based elements down to a minimum. As well as other built-in operations on 
list- and dictionary- data types, for example\texttt{zip} on two arrays to help graph results.

\begin{lstlisting}[language=python,
caption={Example of List Comprehension in Python},
label=lst:python-list-comp]
year = [painting.year for painting in self.paintings 
        if is_absolute(painting.year)]
\end{lstlisting}

Python also boasts a good number of libraries by default; libraries like the \gls{csv} parsing 
library used to read in data. These libraries provide a lot of extended functionality to the 
language. There is also the \gls{pypi} which allows the easy install of 
additional packages.

A lack of experience in Python was a problem to begin with at first and a lot of the early code 
didn't utilise a lot of the useful features Python provides, however after spending time 
contributing to an open source project, also written in Python, these skills quickly developed 
through feedback from commit and being able to read better written source code.

Java was a commonly suggested language, in part due to the wealth of experience there is within
the department. However, on investigation into image processing libraries, it was found that most
of these libraries were either difficult to work with or were not as complete as other available
options.

A recent beta version of \gls{opencv} 2.4.4 provides support for the Java programming language
for desktop user. However, this is vastly more complex to include as part of a project, requiring
additional tools like the Apache Ant built system. The use of a beta library would have run the 
risk of running into bugs in the software and relying on the project to fix them (or writing 
patches and contributing to the project). Due to constrained time it was the right decision not to
use \gls{opencv} with Java.

Java is also very constrained in the nature of typing and lacks a lot of the mathematical features
Python offers (list comprehension, lambda functions, etc.) which helps keep the code-base clean.
This combined with the verbose nature of Java would have increased the size of the project to
less manageable sizes.

Newer \gls{jvm} based languages, Scala for example, could have helped with this problem, but most
of the advantages gained by choosing Java as a language - especially knowledge - would have been
lost in doing so and still lack the mathematical features of Python.

As \gls{opencv} is natively written in C++, it might have made sense to use either C or C++ to write 
this project in and both were strongly considered at the start of the project.

C lacks native object-orientation\footnote{Using structures and functions points it can mimic the 
behaviour to a certain extent, but still lacks inheritance} and is generally difficult to write 
``correctly''. Other issues like cross-platform support were also factors in the choice against C.

C++ has similar problems to C, though it does have more advanced features, including 
object-orientation, which do make it easier to work with, it can still be difficult to produce 
programs without problems. Both C and C++ have a decent set of compilers, from the standard 
\gls{gcc}, which give very basic error helping features, to LLVM based compilers like clang which 
ease the process of finding bugs.

It was decided that, if the project ever managed to produce anything worthy of making into a
library, it would be ported to C++ and investigation would be made into contributing it back into
\gls{opencv}. This would have likely been related to the brush-stroke section of work in the project. 
The only other reason for using C++ or C over Python for this project is if the processing needed
to be completed within a short period of time, as this is a research project it matters very 
little how fast each technique completes in\footnote{And, if a technique does take too long there 
are solutions to this problem through the canning of results, as was performed on \gls{hog} 
analysis (see section~\ref{sec:hog-canning})}.

Newer, more popular languages like Ruby, don't have the support for image processing libraries and
the language features of Ruby don't particularly beat Python's. Many other languages suffer the
same flaws of not having good support for image processing libraries.

Python is not without its flaws, the lack of strictness, especially before running code, does mean
a lot of errors are encountered at runtime. This is a disadvantage of the dynamically typed and
interpreted nature of the language. Statically typed languages are a lot easier to pick up on
typing errors, Python takes a step further than this and uses what is commonly known as duck
typing: ``If it looks like a duck and quacks like a duck, it must be a duck.''\cite{2013Glossary},
this means there's no need for the inheritance hierarchy. 

Not having a compiler to lean on for picking up syntax errors before runtime is another
disadvantage which Python suffers from. However, Python is generally quick enough to run that 
hitting these errors at runtime isn't a huge issue, there are also tools available to help pick up
on these errors. PyLint was used during the project to find syntax errors and clean up code to
typical Python standards, this was useful as most of the code was difficult to test cleanly so
runtime errors were fairly common. The interpreter also has some checking for invalid syntax 
before runtime so it was useful to lean on that on occasion.


\subsubsection{Dependency Management}
setuptools is a Python package designed to allow the easy downloading, building, 
installation, upgrading and uninstallation of Python packages from \gls{pypi}. It can be used to
manage the dependencies of a Python project through the use of a set-up file specifying the 
packages required for a project to run. On build of that project all required packages will be 
downloaded and installed, along with any packages they in turn require.

Again this is a bonus of using Python. Java has very bad dependency management, the only tool
which is widely used to do this is Apache Maven, and is not widely adopted and therefore not all
that useful. \Gls{osgi} is another technology which makes managing dependencies easier, but in 
itself doesn't handle the download and install of packages.

Languages like C and C++ tend to use packages through the operating system or user-based 
compilation, this can lead to a lot of dependency issues if the user is not well versed in the
installation of libraries from source, especially on less common packages which are not provided
through a package manager\footnote{Or on Windows, which has no standardised package manager}.

Newer languages, again using Ruby as an example, do have better dependency management and tend to
be more tied to the language itself, rather than being a tool like setuptools. There aren't
many advantages to this as \gls{pypi} is widely known.


\subsection{Image Processing/Computer Vision Libraries}
There are numerous image processing and/or computer vision libraries available for use, each with
their own flavours and supported programming languages. \gls{opencv} was the eventual choice after a lot
of research into these libraries (as discussed in section~\ref{sec:cv-lib}). \gls{opencv} is one
of the leading computer vision libraries freely available (released under the BSD License) and 
boasts an impressive number of methods and has the benefit of being highly optimised and 
cross-platform.

In comparison to most other computer vision libraries \gls{opencv} feels well thought out and 
polished; with good, well written, documentation for all interfaces (C, C++ and Python). Some of
the Python documentation was a little difficult to use before discovering \gls{cv2}, but this was 
partly because it was no longer being maintained.

Other libraries, notably \gls{fiji}, lacked comprehensive, well laid out documentation. Method 
calls were either not obviously named or buried in some many namespaces that it was impossible to
find the correct call without documentation.


\subsubsection{OpenCV cv2}
Knowing about \gls{cv2} from the start would have been a useful piece of information, the
updates to the library make it a lot easier to use. For example; calling the calculate histogram
method in \gls{opencv} is shown in listings~\ref{lst:cv1-hist}, whilst the same operation in 
\gls{cv2} is shown in listings~\ref{lst:cv2-hist} and is evidently a lot easier to call
and work with.



\begin{lstlisting}[language=python, caption={Creating a Histogram in OpenCV}, label=lst:cv1-hist, 
breaklines=true, frame=single]
import cv

image = cv.LoadImageM("file.png")

planes = [cv.CreateMat(image.rows, image.cols, cv.CV_8UC1) for i in xrange(3)]
planes = planes + [None]

cv.Split(image, planes)

hist = cv.CreateHist([255,255,255], cv.CV_HIST_ARRAY, [[0,255],[0,255],[0,255]], 1)
cv.CalcHist([cv.GetImage(i) for i in planes], hist)
\end{lstlisting}


\begin{lstlisting}[language=python, caption={Creating a Histogram in OpenCV cv2},
label=lst:cv2-hist, breaklines=true, frame=single]
import cv2

image = cv2.imread("file.png")
hist = cv2.calcHist([plane for plane in cv2.split(image)], [0], None, [255,255,255], [0,255,0,255,0,255])
\end{lstlisting}

Another advantage \gls{cv2} has is its tight integration with numpy - a numeric library for Python
and part of the scipy library. This integration allows many more complex operations to be 
performed on analysis results, operations which \gls{opencv} does not provide or provides in ways
only specific to image processing, a notable example of this is being able to take the mean along
an axis to generate the centroid in statistical exemplar classification.

numpy is also useful for non-image processing use; many complex mathematical functions are 
pre-defined which are useful for generating the statistical results for a given set of analysis
and classification techniques after leave-one-out cross validation is performed. The main use was
for Pearson's product-moment correlation coefficient and p-value significance.

Other libraries exist for these functions, but could not reasonably be considered due to the 
integration with \gls{cv2}.
