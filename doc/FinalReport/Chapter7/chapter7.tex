\chapter{Conclusions}
To the best of my knowledge this is the first work that attempts to date work by an artist by 
year. Similarly, it is believed that this is the first attempt to try and perform digital analysis
of paintings from a range of catalogue and web images.

The results presented here show that computer vision \emph{can} help with the job of dating art 
within an artist's body of work. It has been shown that strong, statistically significant 
correlations between a method's allocation of year and the actual year of painting; as well as the
ability to classify 70\% of paintings to within their actual year of painting (within a dataset 
that spans 6 decades). These results are not yet of great use to art historians, but it is hoped 
that future work will be able to improve upon this. Several statistical avenues remain to be 
explored: looking into feature combination and selection, and also investigate the potential for 
treating the year classification problem not as a nearest neighbour problem, but as an ordinal 
regression problem. 

Future directions will also involve testing the methods presented here on the works of other 
artists who have shown great stylistic variation over the course of their career: one plan was to 
build a dataset of, for example, David Hockney works.  Whilst this test has not yet been performed
it is hoped it would be a success: by avoiding brushstroke detection (which we expect to be artist
specific) we hope to have developed techniques with application across a broader range of artistic
styles, and by building techniques which work on catalogue images rather than those captured in 
controlled conditions, there is a better openness to working with paintings from a wider range of 
artists.
