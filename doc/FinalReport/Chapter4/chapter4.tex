\chapter{Implementation}

%The implementation should look at any issues you encountered as you tried to implement your design. During the work, you might have found that elements of your design were unnecessary or overly complex, perhaps third party libraries were available that simplified some of the functions that you intended to implement. If things were easier in some areas, then how did you adapt your project to take account of your findings?

%It is more likely that things were more complex than you first thought. In particular, were there any problems or difficulties that you found during implementation that you had to address? Did such problems simply delay you or were they more significant? Your implementation might well be described in the same chapter as Problems (see below).

\section{Basic Structure}

\subsection{Loading Data}
One of the more key parts to implement before all others in this project is the ability to load in
data from Llyods spreadsheet. The first step of this was to convert it to a \gls{csv} file-type,
which is easier to read programatically.

The initial spreadsheet was slightly different from the version depicted below, Llyod was kind 
enough to update it so it was easier to handle digitally. In this version of the spreadsheet the
filenames were much longer and were in sub-folders depending on their collection. Extra logic was
needed to locate the file (a simple matter of concatenating the collection to the filename as a
directory). The image files in the second version were just flat files which I decided would be
better placed in a data directory.

The move to the newer version was a good excuse to clean up some of the initial code to use better
Python programming practises; replacing messy loops with list comprehension where possible, using
\gls{kwargs} and dictionaries instead of having utility methods, etc.

From the original format show in table~\ref{tab:spreadsheet-format} (see 
table~\ref{tab:spreadsheet} for the full document) the data was converted to the CSV format show
in figure~\ref{fig:csv-spreadsheet}.

\begin{table}[h]
\resizebox{\textwidth}{!}{
\begin{tabular}{|c|c|c|c|c|c|c|c|c|c|c|c|c|} \hline
Filename & ID & Title & Catalogue entry BBC YP & Genre & Height & Width & Area & Materials & 
Collection & image width & image height & image height/image width \\ \hline
001.jpg & 1 & A Chapel in the Tyrol & 1950-1960 & Landscape & 40.7 & 29.8 & 1212.86 & oil on 
hardboard & NLW & 687 & 944 & 1.3741 \\ \hline
\end{tabular}
}
\caption{Layout of the Painting Data Spreadsheet}\label{tab:spreadsheet-format}
\end{table}

\begin{figure}[h]
\resizebox{\textwidth}{!}{
\texttt{001.jpg,1,A Chapel in the Tyrol,1950-1960,Landscape,40.7,29.8,1212.86,oil on hardboard,
NLW,687,944,1.3741}
}
\caption{Painting Data in CSV Format}\label{fig:csv-spreadsheet}
\end{figure}

This \gls{csv} was then parsed using the Python 2.7 in-built \texttt{csv} module with relative 
ease. The example code (see listing~\ref{lst:csv_example_code}) helped with correct resource 
management as, at this point, my Python knowledge was fairly low.


\subsection{Top-Level Classes}
Being used to Java, an initial instinct of mine was to set up interfaces for the majority of 
top-level classes (namely the \texttt{Analyser} and \texttt{Classifier} classes). However, 
because of Python's duck typing, there wasn't possibly to create these classes as interfaces. I 
decided to create them as abstract stub classes which just held placeholder method definitions 
which needed to be overridden in the concrete sub-classes. This was a slightly pointless exercise 
due to the weak typing, but useful for me to get my head around the architecture.

I also implemented the \texttt{Painting} class and had a method to take an array (from the CSV 
file) and create a new instance of it from this. Later this was changed to use Python 
\gls{kwargs} instead as it made more sense for the job. This change was done during the change to 
the second version of Llyods database.


\subsection{Command Line Interface}
As a research program with lots of different analysis and classification techniques the next step
in setting up the basic architecture was to create a set of command-line arguments which would
switch the functionality of the program. These arguments are depicted in table~\ref{tab:args}.

\begin{table}[h]
\centering
\begin{tabular}{|c|c|c|c|} \hline
Name             & Short Flag & Long Flag         & Description\\\hline
Analyser         & \verb+-a+  & \verb+--analyser  +& Switch the analysis technique\\\hline
Machine Learning & \verb+-m+  & \verb+--ml        +& Switch the machine learning technique\\\hline
Data             & \verb+-d+  & \verb+--csv       +& The data file to use\\\hline
GUI              & \verb+-g+  & \verb+--gui       +& Switch the GUI visualisation\\\hline
Binning          & \verb+-5+  & \verb+--bin-years +& Put paintings in 5 year bins\\\hline
Export           & \verb+-e+  & \verb+--export    +& Export analysed data\\\hline
\end{tabular}
\caption{Command Line Arguments}\label{tab:args}
\end{table}

The \texttt{argparse} library handles this nicely, the example code shown in 
listing~\ref{lst:argparse_example_code} shows the usage of this library.


\section{Colour Space Analysis}
Colour space analysis involves performing statistical analysis on different colour models 
(\gls{rgb}, \gls{hsv}, etc.). This gives a very simplistic view of the entire image.

OpenCV offers the \texttt{Avg} method to perform the average across the image, however with a 
further look into the documentation there is also the \texttt{AvgStd} method which performs both 
mean and standard deviation on an image.

The analysed data was just the tuple returned by the \texttt{AvgStd} method. The distance measure
was defined to be the sum of all elements in the tuple (in the case of an \gls{rgb} colour model
the mean red, green and blue and the standard deviation of red, green and blue).

\subsection{Colour Models}
There are many colour models to consider with digital image processing. \Gls{rgb} is one of the
better know colour spaces as it is often how images are captured. It does have a problem in that
all three values can change when the brightness changes.

As one of the main principals of this project is that Kyffin Williams' work darkened over time, it
should follow that \gls{rgb} may not be the best colour model to use.

To account for this it was decided to also use a \gls{hsv} colour model to compare and contrast to
\gls{rgb}.

OpenCV handles colour spaces slightly oddly. Initially it uses \texttt{LoadImageM} to load the 
image, which uses an integer argument as a flag to define whether the image should be loaded in 
colour or grayscale.

From this image you then can use \texttt{CvtColor} to convert the colour model of an image, which 
uses an integer argument as a flag to define a number of different colour spaces.

Once converted, all methods act exactly the same as they would on a \gls{rgb} image.

\subsection{Colour Histograms}
Colour Histograms are a representation of the distribution of colour across an image; this makes
them very powerful for analysing the colour space of an image.

OpenCV provides a number of methods for both calculating and operating upon colour histograms. 
Here the example code (figure~\ref{lst:opencv_histogram}) from the OpenCV documentation was a
useful reference as some of the details of creating histograms in OpenCV are not noted implicitly
in the python documentation.

The distance measure between colour histograms is also handled by OpenCV using a compare histogram
method. Under the covers this uses a Chi Squared (equation~\ref{eq:chi_squared}) method to
compare to histograms of equal dimensions.

\begin{equation}\label{eq:chi_squared}
d(H_1,H_2) =\sum_I{\frac{(H_1(I) - H_2(I))^2}{H_1(I) + H_2(I)}}
\end{equation}


\section{Texture Analysis}

\subsection{Edge Orientation}
\subsubsection{Histogram of Edge Orientation}


\section{Brush-Stroke Analysis}


\section{Classification and Validation}

\subsection{$k$-Nearest Neighbour}

\subsection{Leave-One-Out Cross Validation}

\subsection{Weka 3}
\subsubsection{Attribute-Relation File Format (ARFF)}

\subsection{Exemplars}
\subsubsection{Nearest Exemplar Classification}
To implement Nearest Exemplar Classification was a fairly easy task: Llyod (with help from members
of the \gls{nlw}) provided a secondary spreadsheet which contained all the necessary information 
of exemplar by year (see table~\ref{tab:exemplar-spreadsheet} for the full document).

The spreadsheet was arranged in the format described in table~\ref{tab:exemplar-layout}, from
there it was a simple matter of saving the spreadsheet as a \gls{csv} file and taking some of the
existing code for parsing \gls{csv} files. This caused a slight problem in that the parsed data
didn't have enough information to create a full \texttt{Painting} object, yet all the analysis
techniques worked from these objects.

\begin{table}[h]
\centering
\begin{tabular}{|c|c|c|c|c|} \hline
Filename & ID  & Title                      & Catalogue Entry & Year \\\hline
154.jpg  & 154 & Landscape at Llanaelhaearn & 1947            & 1947 \\\hline
\multicolumn{5}{|c|}{\textit{etc.}}\\\hline
\end{tabular}
\caption{Layout of the Exemplar Spreadsheet}\label{tab:exemplar-layout}
\end{table}

This was solved easily thanks to Python's dynamic typing. A simple class which implemented all the
necessary elements of \texttt{Painting} could be passed to the analysis techniques without any
complaints. With a statically typed language this would have been harder to complete, but there
would have been ways around using sub-classes and so on.

With the exemplars loaded and analysed, the program could continue as normal, until the
classification step.

The idea of Nearest Exemplar Classification is to classify the unknown example using the nearest
exemplar to that example in the feature space. This acts as a $k$-Nearest Neighbour with $k=1$ and
the space of neighbours only including the exemplars, rather than every other example. The 
psuedocode for this is shown in figure~\ref{fig:nec-psuedo}.

Initially this was implemented so that the examples that were exemplars were also classified, but
this is a pointless exercise which only skews the results. Additional logic was added to skip any
example which was an exemplar itself.

\begin{figure}[h]
\begin{algorithmic}
\Function{NearestExemplarClassification}{$example$, $exemplars$}
\If{$example \in exemplars$}
\State \Return $\varnothing$
\EndIf
\State \Return \Call{Nearest}{$example, exemplars$}
\EndFunction
\end{algorithmic}
\caption{Nearest Exemplar Classification Psuedocode}\label{fig:nec-psuedo}
\end{figure}

This proved to give slightly worse correlation per technique than $k$-Nearest Neighbour. This 
result is to be expected; for a start an artistically classified exemplar is unlikely to be the
same as a statistically classified exemplar (see section~\ref{sec:sce}). Also, with fewer examples
to classify against, any variance in the data set (of which there is a lot) will likely be 
magnified.

Lastly, a painting may be picked as an exemplar by an expert for different reasons than any
analysis technique that currently exists can give; emotional connections and knowledge of the
artists history can be very subjective and may not relate to anything put down in paint.


\subsection{Statistically Classified Exemplars}\label{sec:sce}

Another approach to exemplars is to work out a theoretical exemplar for a given period; the 
centroid of paintings within the given feature space for a single year, for example.

The simplest way of working this out is showing in figure~\ref{fig:sce-psuedo}, this works by 
taking the mean (equation~\ref{eq:mean}) of each feature in the set of paintings for a single 
year, this will give the point in feature space that is most central. This is the same technique
used to generate centroids in a clustering algorithm ($k$-Means Clustering, for example).

\begin{figure}[h]
\begin{algorithmic}
\Function{StatisticalClassifyExemplar}{$examples$}
\ForAll{$example \in examples$}
\ForAll{$feature \in example_{features}$}
\State $average_{feature} \gets average_{feature} + example_{feature}$
\EndFor
\EndFor
\ForAll{$feature \in average$}
\State$average_{feature} \gets \frac{average_{feature}}{\Call{Length}{examples}}$
\EndFor
\State \Return $average$
\EndFunction
\end{algorithmic}
\caption{Psuedocode for generating a Statistically Classified Exemplar}\label{fig:sce-psuedo}
\end{figure}

This may become a long operation depending on how many dimensions the feature space has. A
technique like \gls{pca} may be useful to help cut down the number of dimensions needed that this
algorithm uses.


\section{3\textsuperscript{rd} Party Libraries and Tools}

\subsection{Python}
\subsubsection{Python setuptools}

\subsection{OpenCV}

\subsection{scipy \& numpy}

\subsection{matplotlib}

\subsection{Weka 3}
\subsubsection{liac-arff}

\subsection{git \& github}


