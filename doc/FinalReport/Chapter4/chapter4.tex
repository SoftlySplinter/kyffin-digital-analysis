\chapter{Implementation}

%The implementation should look at any issues you encountered as you tried to implement your design. During the work, you might have found that elements of your design were unnecessary or overly complex, perhaps third party libraries were available that simplified some of the functions that you intended to implement. If things were easier in some areas, then how did you adapt your project to take account of your findings?

%It is more likely that things were more complex than you first thought. In particular, were there any problems or difficulties that you found during implementation that you had to address? Did such problems simply delay you or were they more significant? Your implementation might well be described in the same chapter as Problems (see below).

\section{Colour Space Analysis}
Colour space analysis involves performing statistical analysis on different colour models 
(\gls{rgb}, \gls{hsv}, etc.). This gives a very simplistic view of the entire image.

OpenCV offers the \textit{Avg(CvArr):CvScalar} method to perform the average across the image, 
however with a further look into the documentation there is also the 
\textit{AvgStd(CvArr):(CvScalar,CvScalr)} method which performs both mean and standard deviation
on an image.

The analysed data was just the tuple returned by the \textit{AvgStd} method. The distance measure
was defined to be the sum of all elements in the tuple (in the case of an \gls{rgb} colour model
the mean red, green and blue and the standard deviation of red, green and blue).

\subsection{Colour Models}
There are many colour models to consider with digital image processing. \Gls{rgb} is one of the
better know colour spaces as it is often how images are captured. It does have a problem in that
all three values can change when the brightness changes.

As one of the main principals of this project is that Kyffin Williams' work darkened over time, it
should follow that \gls{rgb} may not be the best colour model to use.

To account for this it was decided to also use a \gls{hsv} colour model to compare and contrast to
\gls{rgb}.

OpenCV handles colour spaces slightly oddly. Initially it uses \textit{LoadImageM(str, int):CvMat}
to load the image, where the \textit{int} is a flag to define whether the image should be loaded
in colour or grayscale.

From this image you then can use \textit{CvtColor(CvArr, CvArr, int)} to convert the colour model
of an image. The \textit{int} is a flag to define a number of different colour spaces.

Once converted, all methods act exactly the same as they would on a \gls{rgb} image.


\subsection{Colour Histograms}



\section{Texture Analysis}

\subsection{Edge Orientation}
\subsubsection{Histogram of Edge Orientation}


\section{Brush-Stroke Analysis}


\section{Classification and Validation}

\subsection{K-Nearest Neighbour}

\subsection{Leave-One-Out Cross Validation}

\subsection{Weka 3}
\subsubsection{Attribute-Relation File Format (ARFF)}

\subsection{Exemplars}
\subsubsection{Nearest Exemplar Classification}
To implement Nearest Exemplar Classification was a fairly easy task: Llyod (with help from members
of the \gls{nlw}) provided a secondary spreadsheet which contained all the necessary information 
of exemplar by year (see figure~\ref{fig:exemplar-spreadsheet} for the full document).

The spreadsheet was arranged in the format described in figure~\ref{fig:exemplar-layout}, from
there it was a simple matter of saving the spreadsheet as a \gls{csv} file and taking some of the
existing code for parsing \gls{csv} files. This caused a slight problem in that the parsed data
didn't have enough information to create a full \verb+Painting+ object, yet all the analysis
techniques worked from these objects.

This was solved easily thanks to Python's dynamic typing. A simple class which implemented all the
necessary elements of \verb+Painting+ could be passed to the analysis techniques without any
complaints. With a statically typed language this would have been harder to complete, but there
would have been ways around using sub-classes and so on.

With the exemplars loaded and analysed, the program could continue as normal, until the
classification step.

The idea of Nearest Exemplar Classification is to classify the unknown example using the nearest
exemplar to that example in the feature space. This acts as a $k$-Nearest Neighbour with $k=1$ and
the space of neighbours only including the exemplars, rather than every other example. The 
psuedocode for this is shown in figure~\ref{fig:nec-psuedo}.

Initially this was implemented so that the examples that were exemplars were also classified, but
this is a pointless exercise which only skews the results. Additional logic was added to skip any
example which was an exemplar itself.

\begin{figure}[h]
\begin{algorithmic}
\Method{NearestExemplarClassification}{examples, exemplars}
\end{algorithmic}
\caption{Nearest Exemplar Classification Psuedocode}\label{fig:nec-psuedo}
\end{figure}

\subsubsection{Theoretical Exemplars}


\section{3\textsuperscript{rd} Party Libraries and Tools}

\subsection{Python}
\subsubsection{Python setuptools}

\subsection{OpenCV}

\subsection{scipy \& numpy}

\subsection{matplotlib}

\subsection{Weka 3}
\subsubsection{liac-arff}

\subsection{git \& github}


