\chapter{Background \& Objectives}

%This section should pick-up material from your progress report and enhance it based on the feedback and also your additional experience up to now. 

\section{Sir John ``Kyffin'' Williams}

Sir John ``Kyffin'' Williams (1918-2006) was a Welsh painter and printmaker, widely regarded as 
the defining artist of Wales during the 20\textsuperscript{th} century\cite{Davies2008Welsh}. He was advised to take up
art by a doctor after failing a British Army medical examination because of an `abnormality' 
(epilepsy) as something which would not tax his brain.

He studied at the Slade School of Fine Art and taught art in Highgate School, after which he 
retired to Anglesey until he died in 2006 after a long battle with cancer.

His most characteristic pictures are of Welsh landscapes, painted with thick layers of oil paint
applied with a palette knife\cite{Chilvers2009Dictionary}. Most of his paintings are highly textural; to the point of being
3-dimensional.

As his life progressed Kyffin's `abnormality' grew steadily worse, especially when exposed to 
bright light. As a result most of his paintings are of overcast Welsh landscapes and tend to 
become visibly darker over time\cite{Harris2011How}. By eye it is generally quite easy to approximate the time period
in which a painting was created.

In 1969 he won a scholarship to study and paint in Y Wladfa; the Welsh settlement in Patagonia.
This period of his life is very obvious from his paintings as there is a complete contrast in 
colour between Patagonian and Welsh landscapes.


\section{Interdisciplinary work with the National Library of Wales}

This project was initially suggested through a conversation between Hannah Dee and Gareth ``Llyod''
Roderick about image processing and art. Llyod is a PhD student at the National Library of Wales 
(NLW) researching (TODO: Find out what Llyod's thesis title is). Their initial idea was to be able
to geolocate a Kyffin painting on a map to build up a geographical representation of Kyffin's 
work.

Hannah started to create a prototype for performing geographical analysis, this proved to be a 
difficult task and one which is still being researched.

However, the nature of Kyffin's illness and painting style allows for a second form of analysis:
temporal. As previously stated it is fairly easy to judge by eye a good approximation of the 
period in which a Kyffin painting was created. It should, therefore, follow that this process can
be performed digitally.

When I started this project I was given a ``database'' (in reality this was just a spreadsheet) 
Llyod had produced, containing information of Kyffin Williams' paintings, including: title, year,
category (landscape, portrait, etc.), canvas size and a few additional details which aren't so 
relevant to the project.

The first meeting held was between Llyod, Hannah and I, in which we discussed the current state of
the project, what our aims for the project were and what form of help Llyod could provide to us.
As one of the objectives of this project is to, eventually, get a paper published, the relevant
details of the process we would need to go through if we wanted to do so.

The second meeting was between Hannah, Llyod, Lorna M. Hughes (Llyod's supervisor) and I. Again we
discussed the state of the project. Llyod had also produced a better version of his ``database'' 
to be more machine readable and succinct. A lot of information came from this meeting;

\begin{itemize}
\item The ``cut-off'' point between early and late is around 1973.
\item The size of the canvas might be a useful data point to use in classification, as Kyffin sold
more paintings he would have had the money available for larger canvases and the paint for said
canvas.
\item It is a little dubious as to whether some dates can be trusted. One painting owned by the
NLW was stated to be his last painting, but Lorna believes it was painted much earlier and claimed
to be his last to improve the sale price.
\item Llyod may have found date markings on some paintings. These again may not be accurate, but
may prove to increase the sample size.
\item It should be easy to provide a ``no later than'' estimate for each painting from the art
historians.
\item Paul (?) should be able to produce some exemplars for us as a ground truth.
\item Llyod may be able to find more paintings in the hands of private collectors to increase the 
sample size.
\item Llyod had been playing around with ImageJ to do some basic graph plotting. This might be 
useful to look at further to expand my own work.
\end{itemize}

There were also more detailed discussions about publications, particularly in a digital humanities
journal.


\subsection{Continuation of the Kyffin Project}

There are several projects that could continue on from the Kyffin Project.

One was to use the Learning/Teaching development fund to produce a web-based front-end for of some
of my analysis.

Another venture was to look into PhD funding to build up a 3D map of some of Kyffin's paintings
and being able to display it (perhaps via HTML5 and WebGL) so they can explore the painting 
digitally how it is meant to be in real life.



\section{Existing Work}

\subsection{Edge-Orientated Gradients}\label{sec:existing-hogs}

\subsection{Brush-stroke Analysis}\label{sec:existing-brushstroke}


\section{Analysis Objectives}
Analysis is one of the biggest sections of this project and involves creating techniques which 
will allow comparison of paintings in a way which will allow some form of classification to be
performed on them.

Typically I would expect this to produce some form of high-dimension state space in which each
painting is a point in the state space. From this state space the distance between one painting
and another can be easily resolved using a distance measure like Manhattan distance 
\eqref{eq:manhattan_distance}, euclidean distance \eqref{eq:euclidean_distance} or a distance 
measure more specific to the state space should it be needed (e.g.: chi-squared for histograms).

\begin{equation}\label{eq:manhattan_distance}
d_1(\mathbf{p},\mathbf{q}) = \|\mathbf{p}-\mathbf{q}\|_1 = \sum^{n}_{i=0}{|p_i-q_i|}
\end{equation}

\begin{equation}\label{eq:euclidean_distance}
d_1(\mathbf{p}, \mathbf{q}) = \sqrt{\sum^{n}_{i=0}({q_i-p_i})^2}
\end{equation}

\subsection{Colour-space Analysis}
The simplest way of analysing a digital image is to look at the colours which it consists of.
Doing this is relatively simple; each pixel has a set of values defining the colour of that point,
getting something meaningful from this is less simple.

The simplest strategy is to perform some form of statistical analysis on each painting then use
this for classification. Several good and computationally cheap options exist for this; 
mean \eqref{eq:mean} and standard deviation \eqref{eq:std_dev}, are some good
examples which often come predefined in image processing and computer vision libraries.

\begin{equation}\label{eq:mean}
\upmu = \frac{1}{N}\sum_{i=1}^{N}x_i
\end{equation}

\begin{equation}\label{eq:std_dev}
\sigma = \sqrt{\frac{1}{N}\sum_{i=1}^{N}(x_i - \upmu)^2}
\end{equation}

The representation of colour is another important factor, an RGB representation will have all 
three values change if there are many changes in brightness of the colours whilst a HSV 
representation will only have a single value change.

Therefore, an object of this section should be to explore different colour models and statistical
methods which can be applied to them.

Another useful technique which should be investigated early into the project are image histograms.
These histograms plot the distribution of colour across an image and are therefore a very powerful
method of analysing an image, especially for comparison. As with statistical analysis, histograms
will be largely effective by colour model.

\subsection{Texture Analysis}
As Kyffin Williams' work is very textural, it follows that a main part of the analysis should
focus around the texture of his paintings. Unfortunately for this section, it seems unlikely that
I will be able to get any 3-dimensional models of Kyffin's paintings. This would have been a nice,
if rather large, section of the project.

Instead it is more sensible to look at the orientation of edges in Kyffin's work. Some useful 
pre-existing techniques have already been discussed in section~\ref{sec:existing-hogs}. Histograms
of edge orientation\cite{Dalal2005Histograms} seem like a promising concept which may prove 
relatively simple to implement.

This section may also help with any work into brush-stroke analysis (see 
section~\ref{sec:analysis-brushstroke}).

\subsection{Brush-stroke Analysis}\label{sec:analysis-brushstroke}
With Kyffin's distinctive style and how obviously this style changes over time, the ultimate aim 
of this project is to be able to analyse the brush-strokes\footnote{A slight misnomer as Kyffin 
used a palette knife to paint with rather than a traditional brush} in a painting.

From looking at the paintings it is very apparent that in his earlier work he made a lot more 
strokes than in his later works\footnote{Although this isn't quite true as the canvases he worked
on in his later life tended to be larger}. The strokes in his later work tend to have larger areas
and span more of the canvas.

If it is possible to calculate a rough amount and size of strokes made in a given painting it 
should be a reasonable piece of data to classify on. As previously discussed in section~
\ref{sec:existing-brushstroke} there has already been a decent amount of research into determining
brush-strokes in a painting. 

It would be preferable to try and take one of the techniques discussed in that research and change
it to suit the needs of the project rather than attempting to create a whole new method of 
brush-stroke recognition.

\subsection{Ensemble Techniques}
With some of the aforementioned analysis techniques it makes sense to combine two or more 
techniques together; a good example would be colour histograms and histograms of edge orientation.

This form of analysis is inspired by the concept of the same name in statistics and machine 
learning which tend to obtain better predictive performance. It may also be worth while trying to
weight different techniques so that the techniques which give the best performance affect the 
result of the ensemble technique more.


\section{Classification Objectives}
The overall objective of classification is to be able to label a painting by Kyffin Williams as 
being painted in a given year based on analysis performed on all other paintings with known years.

This ties in with the main aim of this project of being able to classify any Kyffin Williams
painting, whether it has a known or unknown year, as being from a given year. Evidently for 
paintings with an unknown year it is difficult to know how accurately the system has been, so, for
the most part, these paintings have been ignored and those paintings with a known year have made
up the training and validation set.

Because of the small size of paintings with known years it should be computationally viable to 
perform leave-one-out cross validation (figure~\ref{fig:loocv}).

\begin{figure}[h]
\begin{algorithmic}
\Function{LOOCV}{$data$} \Comment{$data$ is a set of all data points}
  \ForAll{$item \in data$}
    \State $classified_{item} \gets$ \Call{classify}{$item, data \setminus \left\{{item}\right\}$} 
  \EndFor
  \Statex
  \Return $classified$
\EndFunction
\end{algorithmic}
\caption{Pseudocode for Leave-One-Out Cross Validation}\label{fig:loocv}
\end{figure}

This can be used to evaluate the performance of the analysis technique and classification 
algorithm. Pearson's product-moment correlation coefficient \eqref{eq:pearsons} between actual year
and classified year has been suggested to be a good performance measure for this project.

\begin{equation}\label{eq:pearsons}
\rho_{X,Y}={\mathrm{cov}(X,Y) \over \sigma_X \sigma_Y} ={E[(X-\mu_X)(Y-\mu_Y)] \over \sigma_X\sigma_Y}
\end{equation}


\subsection{Classification}
One of the simplest methods of classification is $k$-Nearest Neighbour (figure~\ref{fig:knn}) from
this one can take a poll of the years for each neighbour and assign the year of the painting to
classify to be the average of these years.

Depending which form of average you take (mathematical mean \eqref{eq:mean}, median or mode) will
alter the result; although it should be noted that median is very unlikely to give a result on its
own due to the sparseness of the data.

\begin{figure}[h]
\begin{algorithmic}
\Require{$0 < k \leq \left|{data}\right|$} \Comment{$data$ is a set of all data points}
\Function{KNearestNeighbour}{$k, data$}
  \For{$i = 1 \to k$}
    \State $nn_i \gets \Call{Nearest}{data}$
    \State $data \gets data \setminus \left\{{nn_i}\right\}$
    \State $i = i + 1$
  \EndFor
  \Statex
  \Return $nn$
\EndFunction
\end{algorithmic}
\caption{Pseudocode for $k$-Nearest Neighbour}\label{fig:knn}
\end{figure}

\subsubsection{Use of Weka}

\subsubsection{Learning Classifier Systems (LCS)}

\subsection{Exemplars}



