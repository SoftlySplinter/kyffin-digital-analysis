\chapter{Background \& Objectives}

%This section should pick-up material from your progress report and enhance it based on the feedback and also your additional experience up to now. 

\section{Sir John ``Kyffin'' Williams}

Sir John ``Kyffin'' Williams (1918-2006) was a Welsh painter and printmaker, widely regarded as 
the defining artist of Wales during the 20\textsuperscript{th} century\cite{Davies2008Welsh}. He was advised to take up
art by a doctor after failing a British Army medical examination because of an `abnormality' 
(epilepsy) as something which would not tax his brain.

He studied at the Slade School of Fine Art and taught art in Highgate School, after which he 
retired to Anglesey until he died in 2006 after a long battle with cancer.

His most characteristic pictures are of Welsh landscapes, painted with thick layers of oil paint
applied with a palette knife\cite{Chilvers2009Dictionary}. Most of his paintings are highly textural; to the point of being
3-dimensional.

As his life progressed Kyffin's `abnormality' grew steadily worse, especially when exposed to 
bright light. As a result most of his paintings are of overcast Welsh landscapes and tend to 
become visibly darker over time\cite{Harris2011How}. By eye it is generally quite easy to approximate the time period
in which a painting was created.

In 1969 he won a scholarship to study and paint in Y Wladfa; the Welsh settlement in Patagonia.
This period of his life is very obvious from his paintings as there is a complete contrast in 
colour between Patagonian and Welsh landscapes.


\section{Interdisciplinary work with the National Library of Wales}

This project was initially suggested through a conversation between Hannah Dee and Gareth ``Llyod''
Roderick about image processing and art. Llyod is a PhD student at the National Library of Wales 
(NLW) researching (TODO: Find out what Llyod's thesis title is). Their initial idea was to be able
to geolocate a Kyffin painting on a map to build up a geographical representation of Kyffin's 
work.

Hannah started to create a prototype for performing geographical analysis, this proved to be a 
difficult task and one which is still being researched.

However, the nature of Kyffin's illness and painting style allows for a second form of analysis:
temporal. As previously stated it is fairly easy to judge by eye a good approximation of the 
period in which a Kyffin painting was created. It should, therefore, follow that this process can
be performed digitally.

When I started this project I was given a ``database'' (in reality this was just a spreadsheet) 
Llyod had produced, containing information of Kyffin Williams' paintings, including: title, year,
category (landscape, portrait, etc.), canvas size and a few additional details which aren't so 
relevant to the project.

The first meeting held was between Llyod, Hannah and I, in which we discussed the current state of
the project, what our aims for the project were and what form of help Llyod could provide to us.
As one of the objectives of this project is to, eventually, get a paper published, the relevant
details of the process we would need to go through if we wanted to do so.

The second meeting was between Hannah, Llyod, Lorna (Llyod's supervisor) and I. Again we discussed
the state of the project. Llyod had also produced a better version of his ``database'' to be more
machine readable and succinct. A lot of information came from this meeting;

\begin{itemize}
\item The ``cut-off'' point between early and late is around 1973.
\item The size of the canvas might be a useful data point to use in classification, as Kyffin sold
more paintings he would have had the money available for larger canvases and the paint for said
canvas.
\item It is a little dubious as to whether some dates can be trusted. One painting owned by the
NLW was stated to be his last painting, but Lorna believes it was painted much earlier and claimed
to be his last to improve the sale price.
\item Llyod may have found date markings on some paintings. These again may not be accurate, but
may prove to increase the sample size.
\item It should be easy to provide a ``no later than'' estimate for each painting from the art
historians.
\item Paul (?) should be able to produce some exemplars for us as a ground truth.
\item Llyod may be able to find more paintings in the hands of private collectors to increase the 
sample size.
\item Llyod had been playing around with ImageJ to do some basic graph plotting. This might be 
useful to look at further to expand my own work.
\end{itemize}

There were also more detailed discussions about publications, particularly in a digital humanities
journal.


\subsection{Continuation of the Kyffin Project}

There are several projects that could continue on from the Kyffin Project.

One was to use the Learning/Teaching development fund to produce a web-based front-end for of some
of my analysis.

Another venture was to look into PhD funding to build up a 3D map of some of Kyffin's paintings
and being able to display it (perhaps via HTML5 and WebGL) so they can explore the painting 
digitally how it is meant to be in real life.



\section{Existing Work}

% Give a broad view of some of the related work to this project.

\subsection{Edge-Orientated Gradients}

% Discuss the Dalal and Triggs paper on Histograms of Edge-Orientated Gradients and any other papers I've looked at around this area.

\subsection{Brush-stroke Analysis}

% Discuss in detail some of the existing work into brush-stroke analysis. 


\section{Analysis Objectives}

\subsection{Colour-space Analysis}

\subsection{Texture Analysis}

\subsection{Brush-stroke Analysis}


\section{Classification Objectives}

\subsection{Classification}

\subsubsection{Use of Weka}

\subsubsection{Learning Classifier Systems (LCS)}

\subsection{Exemplars}



