\chapter{Background \& Objectives}

%This section should pick-up material from your progress report and enhance it based on the feedback and also your additional experience up to now. 

\section{Sir John ``Kyffin'' Williams}

Sir John ``Kyffin'' Williams (1918-2006) was a Welsh painter and printmaker, widely regarded as 
the defining artist of Wales during the 20\textsuperscript{th} century\cite{Davies2008Welsh}. He was advised to take up
art by a doctor after failing a British Army medical examination because of an `abnormality' 
(epilepsy) as something which would not tax his brain.

He studied at the Slade School of Fine Art and taught art in Highgate School, after which he 
retired to Anglesey until he died in 2006 after a long battle with cancer.

His most characteristic pictures are of Welsh landscapes, painted with thick layers of oil paint
applied with a palette knife\cite{Chilvers2009Dictionary}. Most of his paintings are highly textural; to the point of being
3-dimensional.

As his life progressed Kyffin's `abnormality' grew steadily worse, especially when exposed to 
bright light. As a result most of his paintings are of overcast Welsh landscapes and tend to 
become visibly darker over time\cite{Harris2011How}. By eye it is generally quite easy to approximate the time period
in which a painting was created.

In 1969 he won a scholarship to study and paint in Y Wladfa; the Welsh settlement in Patagonia.
This period of his life is very obvious from his paintings as there is a complete contrast in 
colour between Patagonian and Welsh landscapes.


\section{Interdisciplinary work with the National Library of Wales}

This project was initially suggested through a conversation between Hannah Dee and Gareth ``Llyod''
Roderick about image processing and art. Llyod is a PhD student at the National Library of Wales 
(NLW) researching (TODO: Find out what Llyod's thesis title is). Their initial idea was to be able
to geolocate a Kyffin painting on a map to build up a geographical representation of Kyffin's 
work.

Hannah started to create a prototype for performing geographical analysis, this proved to be a 
difficult task and one which is still being researched.

However, the nature of Kyffin's illness and painting style allows for a second form of analysis:
temporal. As previously stated it is fairly easy to judge by eye a good approximation of the 
period in which a Kyffin painting was created. It should, therefore, follow that this process can
be performed digitally.




\subsection{Future Work}

% Discuss in more detail the future work which could come from this project (summer job, PhD research, etc.)


\section{Existing Work}

% Give a broad view of some of the related work to this project.

\subsection{Edge-Orientated Gradients}

% Discuss the Dalal and Triggs paper on Histograms of Edge-Orientated Gradients and any other papers I've looked at around this area.

\subsection{Brush-stroke Analysis}

% Discuss in detail some of the existing work into brush-stroke analysis. 


\section{Analysis Objectives}

\subsection{Colour-space Analysis}

\subsection{Texture Analysis}

\subsection{Brush-stroke Analysis}


\section{Classification Objectives}

\subsection{Classification}

\subsubsection{Use of Weka}

\subsubsection{Learning Classifier Systems (LCS)}

\subsection{Exemplars}



