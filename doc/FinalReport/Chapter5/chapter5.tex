\chapter{Testing}

%Detailed descriptions of every test case are definitely not what is required here. What is important is to show that you adopted a sensible strategy that was, in principle, capable of testing the system adequately even if you did not have the time to test the system fully.

%Have you tested your system on 'real users'? For example, if your system is supposed to solve a problem for a business, then it would be appropriate to present your approach to involve the users in the testing process and to record the results that you obtained. Depending on the level of detail, it is likely that you would put any detailed results in an appendix.

This chapter details the approach to testing and how cross-validation was
performed to evaluate different analysis and classification techniques.

\section{Overall Approach to Testing}
As this is a research-based project, it is difficult to perform any unit, functional or 
requirements based-testing. Most testing is based on validating the results of analysis and 
classification techniques.

There is, however, a sense in which the validation of analysis techniques is a form of testing.
Higher correlations would suggest that a given analysis technique is both implemented properly and
that it represents the data well.

There is also a sense in which testing that a technique does what it says it will do, is both 
difficult and somewhat unnecessary. One would expect to see the correlation increase as more
complex analysis techniques are implemented, so if a recently implemented technique gives a lower
correlation that a previous technique, then this might be considered as a failing test.

If the correlation is improved by the analysis then, even if the technique isn't working 
completely as advertised, it is still doing something right. It was common for the analysis to be
``tinkered'' with to try and improve the performance of the analysis, so commonly any errors were
picked up at this point.

All analysis techniques were performed across two machines and typically acted similarly. The only
discrepancies noticed were on techniques which heavily relied on floating point numbers. This took the form of 
a few of the correlations being slightly different between the machines. This is believed to be because 
one system has a 32-bit operating system and the other is a 64-bit. This would mean that 
floating point numbers would be slightly more accurate on one machine over the other.


\section{Validation}
Cross-validation is a statistical concept used to assess how the results of statistical analysis
generalise to an independent dataset. Cross-validation is typically applied to machine learning
classifiers to test if a technique is overfitting the data. That is how specific the technique is
to the original data set and how it can cope with new, unseen, examples. For example, a rote learner cannot
classify any unseen examples, as the classification is performed based solely on the training set.

As the overall goal is to be able to classify any Williams painting, it is important that the 
program is able to handle new examples, not just those for which the year is known.

There are many forms for calculating cross-validation, usually involving taking sub-samples of the
original dataset and removing them from the training set to create a validation set.

$k$-fold cross-validation is a common method for performing this. It involves splitting the dataset 
into $k$ sections or \emph{folds}. Each fold is removed from the dataset, the classifier is
trained on the remaining data, then the fold is classified using this trained data. 

The other common form of cross-validation is to take random sub-samples from the dataset and are
removed from the training set and then classified using it. This process is performed repeatedly
until a good number of samples have been taken. This has advantages and disadvantages over $k$-fold
cross-validation, it has the problem that not all data points are considered during validation,
but doesn't depend on the size of the data set.


\subsection{Leave-One-Out Cross-Validation}
Leave-one-out cross-validation is a form of $k$-fold cross-validation, in which the number of folds
is equal to the size of the data set. This leads to the behaviour of leaving a single data point
each time and classifying it against the rest of the data set. Because data sets in machine 
learning are typically very large this usually isn't an option due to time complexity.

However, because of the use of $k$-Nearest Neighbour-based classifiers (which need no re-training 
when data points are added or removed) and a small number of known data points, it is one of the best 
method of cross-validation for this project.

Leave-one-out cross-validation provides us with a classified year for each data point. Because
only data points with known years are used, the correlation between actual and
classified years can be calculated to get a measure of performance.

To produce this value Pearson's product-moment correlation was used (Equation~\ref{eq:pearsons}).
The best techniques would produce a correlation value closer to $1$ than other techniques. Another
part of Pearson's is that it is easy to generate statistical significance (the p-value). This
describes the extent to which the observations could appear by chance in a given single null
hypothesis.

The results of the correlation between actual and classified year is defined to be $r$ 
(formally: $r=\rho_{classified, actual}$). The p-value is defined to be $P(r)$. The ultimate goal 
being to maximise $r$ whilst minimising $P(r)$.

Of course, an exact correlation of $r=1$ would be unrealistic to achieve, especially for such a 
humanities-based project. A good technique might reach correlations of $r \ge 0.5$ and a great
technique might reach correlations of $r \ge 0.75$.


%\subsection{Validation using Weka}

