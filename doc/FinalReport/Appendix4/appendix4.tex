\chapter{Code Samples}\label{sec:code}

\section{Example CSV Parsing Code}
\begin{lstlisting}[language=python,breaklines=true,
caption={Example CSV Parsing Code from http://docs.python.org/2/library/csv.html},
label=lst:csv_example_code, frame=single]
>>> import csv
>>> with open('eggs.csv', 'rb') as csvfile:
...     spamreader = csv.reader(csvfile, delimiter=' ', quotechar='|')
...     for row in spamreader:
...         print ', '.join(row)
Spam, Spam, Spam, Spam, Spam, Baked Beans
Spam, Lovely Spam, Wonderful Spam
\end{lstlisting}

\section{argparse Example Code}
\begin{lstlisting}[language=python,
caption={Example argparse Code from http://docs.python.org/2/library/argparse.html},
label=lst:argparse_example_code, frame=single, breaklines=true]
import argparse

parser = argparse.ArgumentParser(description='Process some integers.')
parser.add_argument('integers',
                   metavar='N',
                   type=int, 
                   nargs='+',
                   help='an integer for the accumulator')
parser.add_argument('--sum',
                   dest='accumulate',
                   action='store_const',
                   const=sum, default=max,
                   help='sum the integers (default: find the max)')

args = parser.parse_args()
print args.accumulate(args.integers)
\end{lstlisting}

\section{Gabor Filter Example Implementation}
\begin{lstlisting}[language=MATLAB,breaklines=true,
label=lst:gabor,
caption={Example implementation of a Gabor Filter in MATLAB\cite{Yang2010Gabor}}, frame=single]
function gb=gabor_fn(bw,gamma,psi,lambda,theta)
% bw    = bandwidth, (1)
% gamma = aspect ratio, (0.5)
% psi   = phase shift, (0)
% lambda= wave length, (>=2)
% theta = angle in rad, [0 pi)

sigma = lambda/pi*sqrt(log(2)/2)*(2^bw+1)/(2^bw-1);
sigma_x = sigma;
sigma_y = sigma/gamma;

sz=fix(8*max(sigma_y,sigma_x));
if mod(sz,2)==0, sz=sz+1;end

% alternatively, use a fixed size
% sz = 60;

[x y]=meshgrid(-fix(sz/2):fix(sz/2),fix(sz/2):-1:fix(-sz/2));
% x (right +)
% y (up +)

% Rotation 
x_theta=x*cos(theta)+y*sin(theta);
y_theta=-x*sin(theta)+y*cos(theta);
 
gb=exp(-0.5*(x_theta.^2/sigma_x^2+y_theta.^2/sigma_y^2)).*cos(2*pi/lambda*x_theta+psi);
\end{lstlisting}


\section{OpenCV Histogram Example Code}


\begin{lstlisting}[language=python,breaklines=true,label=lst:opencv_histogram,
caption={Example Histogram calculation and displaying code from OpenCV\cite{opencv_library}.}, frame=single] 
# Taken from: http://opencv.willowgarage.com/documentation/python/imgproc_histograms.html#calchist
# Calculating and displaying 2D Hue-Saturation histogram of a color image

import sys
import cv

def hs_histogram(src):
    # Convert to HSV
    hsv = cv.CreateImage(cv.GetSize(src), 8, 3)
    cv.CvtColor(src, hsv, cv.CV_BGR2HSV)

    # Extract the H and S planes
    h_plane = cv.CreateMat(src.rows, src.cols, cv.CV_8UC1)
    s_plane = cv.CreateMat(src.rows, src.cols, cv.CV_8UC1)
    cv.Split(hsv, h_plane, s_plane, None, None)
    planes = [h_plane, s_plane]

    h_bins = 30
    s_bins = 32
    hist_size = [h_bins, s_bins]
    # hue varies from 0 (~0 deg red) to 180 (~360 deg red again */
    h_ranges = [0, 180]
    # saturation varies from 0 (black-gray-white) to
    # 255 (pure spectrum color)
    s_ranges = [0, 255]
    ranges = [h_ranges, s_ranges]
    scale = 10
    hist = cv.CreateHist([h_bins, s_bins], cv.CV_HIST_ARRAY, ranges, 1)
    cv.CalcHist([cv.GetImage(i) for i in planes], hist)
    (_, max_value, _, _) = cv.GetMinMaxHistValue(hist)

    hist_img = cv.CreateImage((h_bins*scale, s_bins*scale), 8, 3)

    for h in range(h_bins):
        for s in range(s_bins):
            bin_val = cv.QueryHistValue_2D(hist, h, s)
            intensity = cv.Round(bin_val * 255 / max_value)
            cv.Rectangle(hist_img,
                         (h*scale, s*scale),
                         ((h+1)*scale - 1, (s+1)*scale - 1),
                         cv.RGB(intensity, intensity, intensity), 
                         cv.CV_FILLED)
    return hist_img

if __name__ == '__main__':
    src = cv.LoadImageM(sys.argv[1])
    cv.NamedWindow("Source", 1)
    cv.ShowImage("Source", src)

    cv.NamedWindow("H-S Histogram", 1)
    cv.ShowImage("H-S Histogram", hs_histogram(src))

    cv.WaitKey(0)]
\end{lstlisting}


\section{Computer Vision Research Code}

\subsection{OpenCV}
\begin{lstlisting}[language=Python,
breaklines=true,
caption={Using OpenCV to blur an image},
label=lst:opencv, frame=single]
import cv

im = cv.LoadImageM("tux.png")
blur = cv.CreateMat(im.rows, im.cols, im.type)
cv.Smooth(im, blur, cv.CV_BLUR, 9)
cv.SaveImage("tux_blurred_opencv.png", blur)
\end{lstlisting}

\subsection{cv2}
\begin{lstlisting}[language=Python,
caption={Using cv2 to blur an image},
label=lst:cv2, frame=single]
import cv2

im = cv2.imread("tux.png")
blur = cv2.blur(im, 9)
cv2.imwrite("tux_blurred_cv2.png", blur)
\end{lstlisting}

\subsection{FIJI}
\begin{lstlisting}[language=Java,
breaklines=true,
caption={Using FIJI to blur an image},
label=lst:fiji, frame=single]
package fiji;

import ij.IJ;
import ij.io.FileSaver;
import imagescience.feature.Smoother;
import imagescience.image.ColorImage;
import imagescience.image.Image;

public final class Blur {
	public static final String IMG_PATH = "tux.png";
	public static final String OUTPUT_PATH = "tux_fiji.png";

	public static void main(String[] args) {
		final Image image = new ColorImage(IJ.openImage(IMG_PATH));
		final Smoother s = new Smoother();
		final Image blurred = s.gauss(image, 9f);
		final FileSaver fs = new FileSaver(blurred.imageplus());
		fs.saveAsPng(OUTPUT_PATH);
	}
}
\end{lstlisting}

\subsection{IVT}
\begin{lstlisting}[language=C++,
breaklines=true,
caption={Using IVT to blur an image},
label=lst:ivt, frame=single]
#include <stdio.h>
#include "Image/ByteImage.h"
#include "Image/ImageProcessor.h"

int main(int argc, char **argv) {
	CByteImage in;
	CByteImage out;
	char* from = "tux.png";
	char* to = "tux_ivt.png";

	in.LoadFromFile(from)
	out = CByteImage(in);
	ImageProcessor::GaussianSmooth(&in, &out, 1.0f, 3);
	out.SaveToFile(to)
}
\end{lstlisting}
