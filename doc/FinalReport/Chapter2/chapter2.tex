%\addcontentsline{toc}{chapter}{Development Process}
\chapter{Development Process}

%You need to describe briefly the life cycle model that you used. Do not force your project into the waterfall model if it is better described by prototyping or some other evolutionary model. You do not need to write about all of the different process models that you are aware of. Focus on the process model that you have used. It is possible that you needed to adapt an existing process model to suit your project; clearly identify what you used and how you adapted it for your needs.

%In most cases, the agreed objectives or requirements will be the result of a compromise between what would ideally have been produced and what was felt to be possible in the time available. A discussion of the process of arriving at the final list is usually appropriate.

%You should briefly describe the design method you used and any support tools that you used. You should discuss your choice of implementation tools - programming language, compilers, database management system, program development environment, etc.

This chapter defines the development processes which were used to aid the production this project and how they
were changed to suit the needs. The decisions behind the choices
of development process are also discussed and evaluated.

\section{Introduction}
This project was developed using a mixture of iterative development and rapid prototyping. As a
research project it does not need a heavy-weight development process. Such a process would be
too cumbersome to adapt to the changing nature of research. Agile methodologies would have also 
provided the adaptability required, but are designed for a team-based approach so would have
needed a lot of modification to be applicable to a team of one.

Maintenance and testing are known to be the large majority of focus in the methodology chosen,
experience in the field of software engineering typically focuses an individual on both these 
topics. However, as this was a research project it was unlikely to be used much after the end of 
the project. If it were to be used in further research the author or the supervisor would 
be available to provide assistance. It was decided that maintenance was not an issue for this project
and could be disregarded to a reasonable extent. 

Due to the research nature of this project, testing would prove to be difficult. Proving that a
given technique would run was relatively straight-forward by leaning on the interpreter (or 
compiler in cases of any statically typed elements). However, proving a technique would do what it was
expected to do would take so much time out of the implementation of other techniques, that it would be a 
pointless task. Especially when a lot of the techniques would just call external libraries to
implement the parts that were likely to fail (e.g. applying filters through \gls{opencv}).

With both these two issues being lesser than usual it was easy to discount a lot of the traditional
and formal methods from the list of potential methodologies. Waterfall and Spiral models, for 
example, had too much of a focus on testing and maintenance, and on the management of risk.

Iterative development involves developing a system through repeated cycles, at each iteration 
design modifications are made and new functionality is added. Iterative development encourages 
modular design and implementation, suiting this project well as it will have a number of 
techniques to both analyse and classify data points.

Rapid Prototyping involves quickly producing a working prototype of an area of a system to get a
working implementation of that area. This can then be evolved to improve the design and 
implementation as needed. This fits well as a method for implementing the various techniques in 
the system, they can be rapidly prototyped to get a set of results, then improved to increase the
performance and/or accuracy of the technique.

The requirements for the iterations were typically decided in the weekly project meetings, normally 
taking the form of completing a given technique before the next meeting. The technique
was then implemented as a prototype and any adjustments would be made to the system in line with
the iterative process, including improving existing techniques that might have been earlier 
prototyped in an evolutionary fashion.

Initially there were a set of topics which were aimed to be completed:

\begin{itemize}
\item Colour-space analysis
\item $k$-Nearest Neighbour classification
\item Histogram-based colour-space analysis
\item Texture analysis
\item Brush-stroke analysis
\end{itemize}

Each of the above items had several parts to them; colour-space analysis included different 
colour-modes: \gls{rgb} and \gls{hsv}. However as a research project new ideas were often 
suggested during meetings or as a part of research so this list of requirements grew during the
project.

Python helped a lot with both methodologies, thanks to it's dynamic typing it is easy to deal with
changing forms of data. It also has very readable syntax whilst still remaining compact in terms
of lines of code, this makes locating areas which need changing very simple.

Version control was a strong requirement. Having used git for many personal projects the 
environment was familiar. The decentralised nature of git allowed for the remote working on any
machine with an internet connection, so progress was not delayed with the switching of computers.

For hosting the GitHub service was used, GitHub is one of the most popular git hosting services
available and offers free private repositories for students (GitHub is commonly used for Open 
Source software development so most repositories are public by default). It should be noted that
this service is known to be very stable and reliable, but even during brief moments of downtime,
the decentralised nature of git meant that work could continue regardless.

\section{Modifications}
%Did you have to modify the model to suit a one-person project. If so, what did you change and why? 

The main modifications made to iterative development was to shift focus away from testing and into
evolution and implementation. It was also changed to that each iteration was a rapid prototype,
to allow merging of the two techniques. But the evolution step was focused on the whole system.
A good example of this is when finding out about \emph{cv2} and going back to change all the old
techniques to use the latest version of \emph{OpenCV}.

Rapid prototyping was modified very little. However, as there was no official client and there were only meetings with
Hannah once a week, there was very little client focus. This means it was just used as a tool for rapidly
implementing techniques in a way which could be improved and evolved later on. Although this isn't
exactly in the spirit of rapid prototyping, it does fit fairly well with iterative development.

For a solo project, there wasn't any focus on team collaboration. Though iterative development 
isn't an official agile methodology, it does act quite similarly to one. Therefore it is quite 
tied to collaborating code as part of the iterative cycle. This project didn't take advantage of
any of the collaborative parts, nor any review of coding standards. The only review of code was
of the author's own standards. Whilst these are very high, inexperience into writing Python 
code led to very messy code initially.

The evolutionary approach of the methodology allowed for this code to be cleaned up when
required. Especially when modifying existing code later on in the project, when more knowledge into
the language had been gained.
